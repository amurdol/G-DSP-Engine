% ============================================================================
% G-DSP Engine — Fase 0: Análisis de Sistema y Aritmética de Punto Fijo
% ============================================================================
% Autor  : G-DSP Team
% Target : Gowin GW1NR-LV9QN88PC6/I5 (Tang Nano 9K)
% ============================================================================

\documentclass[11pt, a4paper]{article}

% ============================================================================
% Paquetes
% ============================================================================
\usepackage[utf8]{inputenc}
\usepackage[T1]{fontenc}
\usepackage[spanish]{babel}
\usepackage{amsmath, amssymb, amsfonts}
\usepackage{graphicx}
\usepackage{booktabs}
\usepackage{array}
\usepackage{tabularx}
\usepackage{xcolor}
\usepackage{listings}
\usepackage{hyperref}
\usepackage{geometry}
\usepackage{fancyhdr}
\usepackage{siunitx}
\usepackage{float}

\geometry{margin=2.5cm}

% ============================================================================
% Configuración de listados de código
% ============================================================================
\definecolor{codegreen}{rgb}{0,0.6,0}
\definecolor{codegray}{rgb}{0.5,0.5,0.5}
\definecolor{codepurple}{rgb}{0.58,0,0.82}
\definecolor{backcolour}{rgb}{0.97,0.97,0.97}

\lstdefinestyle{verilog}{
    backgroundcolor=\color{backcolour},
    commentstyle=\color{codegreen},
    keywordstyle=\color{blue},
    numberstyle=\tiny\color{codegray},
    stringstyle=\color{codepurple},
    basicstyle=\ttfamily\footnotesize,
    breaklines=true,
    numbers=left,
    numbersep=5pt,
    frame=single,
    captionpos=b
}

% ============================================================================
% Encabezado y pie de página
% ============================================================================
\pagestyle{fancy}
\fancyhf{}
\lhead{G-DSP Engine}
\rhead{Fase 0: Análisis de Sistema}
\cfoot{\thepage}

% ============================================================================
% Información del documento
% ============================================================================
\title{%
    \textbf{G-DSP Engine} \\[0.5em]
    \Large Fase 0: Definición del Sistema y Análisis de Punto Fijo \\[0.5em]
    \large Trabajo de Fin de Grado — Módem 16-QAM en FPGA
}
\author{G-DSP Team}
\date{\today}

% ============================================================================
% Documento
% ============================================================================
\begin{document}

\maketitle
\tableofcontents
\newpage

% ============================================================================
\section{Introducción}
% ============================================================================

Este documento establece los fundamentos del proyecto G-DSP Engine, un procesador
de banda base 16-QAM implementado íntegramente en lógica digital sobre la FPGA
Gowin GW1NR-LV9. Se definen los parámetros del sistema, el formato aritmético de
punto fijo y las restricciones de recursos que guían todas las decisiones de diseño
en fases posteriores.

La \textbf{Fase 0} comprende:
\begin{itemize}
    \item Selección de la plataforma hardware y sus limitaciones
    \item Definición de parámetros del sistema (frecuencias, tasas de símbolo)
    \item Análisis y selección del formato de punto fijo
    \item Presupuesto de recursos DSP
    \item Establecimiento del modelo de referencia en Python (Golden Model)
\end{itemize}

% ============================================================================
\section{Plataforma Hardware}
% ============================================================================

\subsection{Sipeed Tang Nano 9K}

El proyecto se desarrolla sobre la tarjeta de desarrollo \textbf{Sipeed Tang Nano 9K},
que incorpora la FPGA Gowin GW1NR-LV9QN88PC6/I5. Esta plataforma fue seleccionada
por su equilibrio entre prestaciones, costo y disponibilidad de interfaces
multimedia (HDMI nativo).

\begin{table}[H]
\centering
\caption{Recursos de la FPGA GW1NR-LV9}
\label{tab:fpga_resources}
\begin{tabular}{@{}lrl@{}}
\toprule
\textbf{Recurso}      & \textbf{Cantidad} & \textbf{Notas} \\
\midrule
LUTs (4-input)        & 8,640             & Lógica combinacional \\
Flip-Flops            & 6,480             & Registros \\
DSP Slices (MULT9)    & 20                & Multiplicadores 9$\times$9 \\
BSRAM (18 Kbit)       & 26                & Block RAM \\
PSRAM                 & 64 Mbit           & HyperRAM externa (no utilizada) \\
PLL                   & 2                 & Síntesis de reloj \\
\bottomrule
\end{tabular}
\end{table}

\subsection{Arquitectura DSP}

Los bloques DSP del GW1NR-LV9 son de tipo \textbf{MULT9/pREG}, que proporcionan:
\begin{itemize}
    \item Un multiplicador \textbf{9$\times$9 firmado} nativo
    \item Cascada para formar \textbf{18$\times$18} (2 slices) o \textbf{36$\times$36} (4 slices)
    \item Registro de pipeline integrado (pREG)
\end{itemize}

Para datos de 12 bits, utilizamos la configuración 18$\times$18, consumiendo
2 slices MULT9 por multiplicador completo.

% ============================================================================
\section{Parámetros del Sistema}
% ============================================================================

Los parámetros fundamentales del sistema están centralizados en el paquete
\texttt{gdsp\_pkg.sv}, garantizando consistencia entre todos los módulos RTL.

\begin{table}[H]
\centering
\caption{Parámetros del sistema definidos en \texttt{gdsp\_pkg.sv}}
\label{tab:system_params}
\begin{tabular}{@{}llll@{}}
\toprule
\textbf{Parámetro}     & \textbf{Símbolo}     & \textbf{Valor}      & \textbf{Unidad} \\
\midrule
Frecuencia de reloj    & $f_{clk}$            & 27                  & MHz \\
Reloj de píxel (HDMI)  & $f_{pix}$            & 25.2                & MHz \\
Muestras por símbolo   & SPS                  & 4                   & --- \\
Tasa de muestreo       & $f_s = f_{clk}$      & 27                  & Msps \\
Tasa de símbolo        & $f_{sym} = f_s$/SPS  & 6.75                & MBaud \\
Bits por símbolo       & ---                  & 4                   & bits \\
Tasa de bits           & $R_b$                & 27                  & Mbps \\
\midrule
Factor de roll-off     & $\alpha$             & 0.25                & --- \\
Taps del filtro RRC    & $N_{taps}$           & 5                   & --- \\
\bottomrule
\end{tabular}
\end{table}

\subsection{Cálculo de la Tasa de Símbolo}

Con un reloj de sistema de \SI{27}{\mega\hertz} y 4 muestras por símbolo:
\begin{equation}
    f_{sym} = \frac{f_{clk}}{SPS} = \frac{27 \times 10^6}{4} = 6.75 \text{ MBaud}
\end{equation}

Para modulación 16-QAM ($\log_2(16) = 4$ bits/símbolo):
\begin{equation}
    R_b = f_{sym} \times \log_2(M) = 6.75 \times 4 = 27 \text{ Mbps}
\end{equation}

\subsection{Ancho de Banda Ocupado}

El ancho de banda del filtro RRC con factor de roll-off $\alpha = 0.25$:
\begin{equation}
    BW = (1 + \alpha) \times f_{sym} = 1.25 \times 6.75 = 8.4375 \text{ MHz}
\end{equation}

Este ancho de banda cumple holgadamente con la frecuencia de Nyquist
($f_s/2 = 13.5$ MHz).

% ============================================================================
\section{Formato de Punto Fijo}
% ============================================================================

\subsection{Notación Q$m$.$n$}

Adoptamos la notación \textbf{Q$m$.$n$} donde:
\begin{itemize}
    \item $m$ = número de bits de parte entera (excluyendo signo)
    \item $n$ = número de bits de parte fraccionaria
    \item Bit de signo implícito en complemento a dos
    \item Ancho total: $W = 1 + m + n$ bits
\end{itemize}

\subsection{Selección: Q1.11 (12 bits)}

Tras evaluar múltiples opciones, se selecciona el formato \textbf{Q1.11}:

\begin{table}[H]
\centering
\caption{Comparativa de formatos de punto fijo candidatos}
\label{tab:formats}
\begin{tabular}{@{}ccccc@{}}
\toprule
\textbf{Formato} & \textbf{Bits} & \textbf{Rango} & \textbf{Resolución (LSB)} & \textbf{SQNR} \\
\midrule
Q1.7   & 8  & $[-1, +0.992]$   & $7.81 \times 10^{-3}$ & $\sim$44 dB \\
\rowcolor{green!15}
Q1.11  & 12 & $[-1, +0.9995]$  & $4.88 \times 10^{-4}$ & $\sim$68 dB \\
Q1.15  & 16 & $[-1, +0.99997]$ & $3.05 \times 10^{-5}$ & $\sim$92 dB \\
Q2.10  & 12 & $[-2, +1.999]$   & $9.77 \times 10^{-4}$ & $\sim$62 dB \\
\bottomrule
\end{tabular}
\end{table}

\subsection{Propiedades del Formato Q1.11}

\begin{equation}
    \text{Rango: } \quad V \in \left[-1.0, \; +1.0 - 2^{-11}\right] \approx [-1.0, \; +0.99951]
\end{equation}

\begin{equation}
    \text{Resolución (1 LSB): } \quad \Delta = 2^{-11} \approx 4.88 \times 10^{-4}
\end{equation}

\begin{equation}
    \text{Conversión a entero: } \quad x_{int} = \text{round}(x_{float} \times 2^{11})
\end{equation}

\begin{equation}
    \text{Conversión a flotante: } \quad x_{float} = \frac{x_{int}}{2^{11}}
\end{equation}

% ============================================================================
\section{Análisis de Precisión}
% ============================================================================

\subsection{Rango Dinámico}

El rango dinámico de un formato de punto fijo de $N$ bits se calcula como:
\begin{equation}
    DR = 20 \log_{10}\left(\frac{V_{max}}{V_{min}}\right) = 20 \log_{10}\left(\frac{2^{N-1} - 1}{1}\right)
\end{equation}

Para Q1.11 ($N = 12$):
\begin{equation}
    DR = 20 \log_{10}(2047) \approx 66.2 \text{ dB}
\end{equation}

\subsection{Relación Señal a Ruido de Cuantización (SQNR)}

Para una señal sinusoidal de amplitud completa, la SQNR teórica es:
\begin{equation}
    \boxed{SQNR = 6.02N + 1.76 \text{ dB}}
\end{equation}

Donde $N$ es el número de bits. Para Q1.11:
\begin{equation}
    SQNR = 6.02 \times 12 + 1.76 = 72.24 + 1.76 = 74.0 \text{ dB (teórico)}
\end{equation}

En la práctica, para señales QAM cuya distribución no es sinusoidal pura,
se obtiene aproximadamente:
\begin{equation}
    SQNR_{QAM} \approx 68 \text{ dB}
\end{equation}

\subsection{Margen sobre el Requisito de 16-QAM}

Para 16-QAM con BER = $10^{-5}$, el SNR mínimo requerido es aproximadamente
\textbf{20 dB}. Con 68 dB de SQNR:
\begin{equation}
    \text{Margen} = SQNR - SNR_{req} = 68 - 20 = \boxed{48 \text{ dB}}
\end{equation}

Este margen garantiza que el ruido de cuantización será completamente
despreciable frente al ruido térmico del canal.

% ============================================================================
\section{Justificación de 12 Bits}
% ============================================================================

La selección de 12 bits (Q1.11) responde a un equilibrio óptimo:

\begin{enumerate}
    \item \textbf{Compatibilidad DSP:} Los multiplicadores 18$\times$18 de Gowin
          aceptan nativamente operandos de hasta 18 bits. Con 12 bits:
          \begin{equation}
              \text{Producto} = 12 + 12 = 24 \text{ bits}
          \end{equation}
          El multiplicador produce 36 bits, dejando 12 bits de margen para
          acumulación directa dentro del DSP.

    \item \textbf{Ancho del acumulador FIR:} Para 5 taps:
          \begin{equation}
              W_{accum} = W_{prod} + \lceil \log_2(N_{taps}) \rceil = 24 + 3 = 27 \text{ bits}
          \end{equation}

    \item \textbf{Eficiencia de recursos:} 12 bits vs 16 bits reduce el consumo
          de registros y rutas de interconexión en $\sim$25\%, crítico en una
          FPGA de 9K LUTs.

    \item \textbf{Margen SQNR:} 48 dB sobre el requisito asegura que la
          cuantización no degradará el rendimiento del módem.
\end{enumerate}

% ============================================================================
\section{Niveles de Constelación 16-QAM}
% ============================================================================

La constelación 16-QAM normalizada utiliza niveles $\{\pm 1, \pm 3\}/\sqrt{10}$
en cada eje, garantizando potencia media unitaria.

\begin{table}[H]
\centering
\caption{Niveles 16-QAM en Q1.11}
\label{tab:qam_levels}
\begin{tabular}{@{}cccc@{}}
\toprule
\textbf{Nivel} & \textbf{Valor Flotante} & \textbf{Cálculo} & \textbf{Q1.11 (decimal)} \\
\midrule
$-3/\sqrt{10}$ & $-0.948683$ & $\text{round}(-0.948683 \times 2048)$ & $-1943$ \\
$-1/\sqrt{10}$ & $-0.316228$ & $\text{round}(-0.316228 \times 2048)$ & $-648$  \\
$+1/\sqrt{10}$ & $+0.316228$ & $\text{round}(+0.316228 \times 2048)$ & $+648$  \\
$+3/\sqrt{10}$ & $+0.948683$ & $\text{round}(+0.948683 \times 2048)$ & $+1943$ \\
\bottomrule
\end{tabular}
\end{table}

Estos valores están parametrizados en \texttt{gdsp\_pkg.sv}:
\begin{lstlisting}[style=verilog, caption={Niveles QAM en gdsp\_pkg.sv}]
parameter signed [DATA_WIDTH-1:0] QAM_NEG3 = -12'sd1943;
parameter signed [DATA_WIDTH-1:0] QAM_NEG1 = -12'sd648;
parameter signed [DATA_WIDTH-1:0] QAM_POS1 =  12'sd648;
parameter signed [DATA_WIDTH-1:0] QAM_POS3 =  12'sd1943;
\end{lstlisting}

\begin{figure}[H]
\centering
\includegraphics[width=0.6\textwidth]{../figures/constellation_16qam.png}
\caption{Constelación 16-QAM ideal con codificación Gray. Los 16 símbolos se
         distribuyen en una cuadrícula 4$\times$4 con niveles $\{\pm 648, \pm 1943\}$
         en formato Q1.11, correspondientes a $\{\pm 1, \pm 3\}/\sqrt{10}$ normalizados.}
\label{fig:constellation_ideal}
\end{figure}

% ============================================================================
\section{Presupuesto de Errores de Cuantización}
% ============================================================================

\begin{table}[H]
\centering
\caption{Error de truncamiento por etapa del módem}
\label{tab:error_budget}
\begin{tabular}{@{}lccc@{}}
\toprule
\textbf{Etapa}        & \textbf{Entrada} & \textbf{Salida} & \textbf{Error (LSB)} \\
\midrule
QAM Mapper            & LUT exacta       & 12 bits         & 0 (exacto)           \\
RRC Tx Filter         & 12 bits          & 12 bits         & $\sim$0.5 (redondeo) \\
Suma AWGN             & 12 + 12 bits     & 12 bits         & $\sim$0.5            \\
RRC Rx Filter         & 12 bits          & 12 bits         & $\sim$0.5            \\
Gardner Interp.       & 12 bits          & 12 bits         & $\sim$0.5            \\
Costas Derotator      & 12 bits          & 12 bits         & $\sim$0.5            \\
\midrule
\textbf{Total acum.}  & ---              & ---             & $\sim$3 LSB peor caso \\
\bottomrule
\end{tabular}
\end{table}

El error acumulado de 3 LSB en Q1.11 representa:
\begin{equation}
    \varepsilon_{total} = 3 \times 2^{-11} \approx 1.46 \times 10^{-3}
\end{equation}

Este valor es despreciable comparado con el piso de ruido del canal AWGN
en cualquier escenario práctico de interés.

% ============================================================================
\section{Presupuesto de Recursos DSP}
% ============================================================================

Los 20 slices MULT9 disponibles se distribuyen según:

\begin{table}[H]
\centering
\caption{Asignación de bloques DSP}
\label{tab:dsp_budget}
\begin{tabular}{@{}lcc@{}}
\toprule
\textbf{Bloque}       & \textbf{DSP Slices} & \textbf{Notas} \\
\midrule
RRC Tx (semiplegado)  & 2--4                & Multiplexado temporal \\
RRC Rx (semiplegado)  & 2--4                & Misma arquitectura \\
Gardner interpolador  & 2                   & Interpolación lineal \\
Costas Loop           & 4                   & Mult. complejo (3 reales) \\
AWGN / misc           & 2--4                & Escalado de ruido \\
\midrule
\textbf{Reserva}      & 2--4                & Cierre temporal / debug \\
\midrule
\textbf{TOTAL}        & $\leq$ 20           & \checkmark Cabe \\
\bottomrule
\end{tabular}
\end{table}

% ============================================================================
\section{Modelo de Referencia (Golden Model)}
% ============================================================================

Se desarrolló un modelo de referencia bit-exacto en Python (\texttt{scripts/golden\_model.py})
que implementa toda la cadena de señal con aritmética idéntica al RTL.

\subsection{Funciones del Golden Model}

\begin{itemize}
    \item Generación de coeficientes RRC normalizados para Q1.11
    \item Mapeo 16-QAM con codificación Gray
    \item Filtrado RRC (convolución) con truncamiento idéntico al hardware
    \item Generación de vectores de test en formato \texttt{.hex} y \texttt{.mem}
    \item Plots de constelación, diagrama de ojo y espectro
\end{itemize}

\subsection{Normalización de Coeficientes}

Los coeficientes del filtro RRC se normalizan para evitar desbordamiento:
\begin{equation}
    h_{norm}[n] = \frac{h[n]}{\max|h[n]|} \times (2^{FRAC\_BITS - 1} - 1)
\end{equation}

Esto garantiza que el tap de mayor magnitud sea ligeramente menor que 0.5
en representación Q1.11, previniendo overflow durante la acumulación del FIR.

% ============================================================================
\section{Conclusiones de Fase 0}
% ============================================================================

Esta fase estableció los cimientos del proyecto:

\begin{enumerate}
    \item \textbf{Plataforma validada:} Tang Nano 9K con 20 DSP slices y 8.6K LUTs
          es suficiente para un módem 16-QAM completo.

    \item \textbf{Formato Q1.11:} Proporciona 68 dB de SQNR, 48 dB de margen
          sobre el requisito de 16-QAM, con eficiencia de recursos óptima.

    \item \textbf{Parámetros del sistema:} 27 MHz de reloj, 6.75 MBaud, 27 Mbps,
          $\alpha = 0.25$, 5 taps RRC (limitado por recursos DSP).

    \item \textbf{Golden Model:} Referencia bit-exacta en Python para verificación
          rigurosa del RTL en todas las fases subsiguientes.

    \item \textbf{Salida de video:} HDMI 480p a 25.2 MHz (VGA 640$\times$480 @ 60 Hz)
          para visualización de constelación en tiempo real.
\end{enumerate}

% ============================================================================
\section{Referencias}
% ============================================================================

\begin{enumerate}
    \item Gowin Semiconductor, \textit{GW1NR Series FPGA Products Datasheet},
          DS861, 2023.
    \item R. G. Lyons, \textit{Understanding Digital Signal Processing},
          3rd ed., Prentice Hall, 2011, Ch. 12.
    \item J. G. Proakis and M. Salehi, \textit{Digital Communications},
          5th ed., McGraw-Hill, 2008, Ch. 5.
    \item Xilinx, \textit{Efficient Shift Registers, LFSR Counters, and Long
          Pseudo-Random Sequence Generators}, XAPP 052, 1996.
\end{enumerate}

\end{document}
