% !TEX root = fase1_tx_subsystem.tex
% ============================================================================
% G-DSP Engine — Documentacion Tecnica: Fase 1
% Implementacion del Subsistema de Transmision
% ============================================================================
% Autor      : G-DSP Team
% Proyecto   : TFG — Procesador Banda Base 16-QAM sobre Gowin GW1NR-9
% Compilar   : pdflatex fase1_tx_subsystem.tex  (x2 para referencias)
% ============================================================================

\documentclass[a4paper,11pt,spanish]{article}

% ---------------------------------------------------------------------------
% Paquetes
% ---------------------------------------------------------------------------
\usepackage[utf8]{inputenc}
\usepackage[T1]{fontenc}
\usepackage[spanish,es-tabla]{babel}
\usepackage{amsmath,amssymb,amsfonts}
\usepackage{booktabs}
\usepackage{array}
\usepackage{graphicx}
\usepackage{xcolor}
\usepackage{listings}
\usepackage{siunitx}
\usepackage{hyperref}
\usepackage{geometry}
\usepackage{caption}
\usepackage{subcaption}
\usepackage{enumitem}
\usepackage{float}
\usepackage{tikz}
\usetikzlibrary{arrows.meta, positioning, calc, shapes.geometric}

\geometry{margin=2.5cm}

\sisetup{
    output-decimal-marker = {,},
    per-mode = symbol
}

% ---------------------------------------------------------------------------
% Configuracion de listings para SystemVerilog
% ---------------------------------------------------------------------------
\lstdefinelanguage{SystemVerilog}{
    morekeywords={module, endmodule, input, output, logic, signed, parameter,
                  int, always_ff, always_comb, posedge, negedge, if, else,
                  begin, end, for, case, endcase, function, automatic,
                  endfunction, localparam, assign, typedef, package,
                  endpackage, import, default, generate, endgenerate},
    sensitive=true,
    morecomment=[l]{//},
    morecomment=[s]{/*}{*/},
    morestring=[b]",
}

\lstset{
    language=SystemVerilog,
    basicstyle=\ttfamily\footnotesize,
    keywordstyle=\bfseries\color{blue!70!black},
    commentstyle=\itshape\color{green!50!black},
    stringstyle=\color{red!60!black},
    numbers=left,
    numberstyle=\tiny\color{gray},
    numbersep=5pt,
    frame=single,
    breaklines=true,
    captionpos=b,
    tabsize=4,
    showstringspaces=false,
}

% ---------------------------------------------------------------------------
% Datos del documento
% ---------------------------------------------------------------------------
\title{%
    \textbf{G-DSP Engine} \\[0.5em]
    \Large Implementaci\'{o}n del Subsistema de Transmisi\'{o}n \\[0.3em]
    \large Fase~1 --- Documentaci\'{o}n T\'{e}cnica del TFG
}
\author{G-DSP Team}
\date{Febrero 2026}

% ============================================================================
\begin{document}
% ============================================================================

\maketitle
\tableofcontents
\newpage

% ============================================================================
\section{Introducci\'{o}n al Subsistema de Transmisi\'{o}n}
\label{sec:intro_tx}
% ============================================================================

El subsistema de transmisi\'{o}n (TX) del procesador banda base G-DSP Engine
constituye la primera etapa funcional del m\'{o}dem 16-QAM implementado
sobre la FPGA Gowin GW1NR-LV9QN88PC6/I5 (placa de desarrollo Sipeed Tang
Nano~9K). Su funci\'{o}n es generar una se\~{n}al banda base conformada
espectralmente, lista para su transmisi\'{o}n por el canal o, en el \'{a}mbito
de este proyecto, para su procesamiento por el subsistema de recepci\'{o}n
y su visualizaci\'{o}n en tiempo real mediante HDMI.

La cadena de transmisi\'{o}n implementada sigue el diagrama de bloques
cl\'{a}sico de un transmisor QAM digital:

\begin{equation}
    \text{PRBS-23} \;\xrightarrow{\;4\;\text{bits}}\;
    \text{Mapper 16-QAM} \;\xrightarrow{\;I/Q\;}\;
    \uparrow\!{}_{\times 4} \;\xrightarrow{\;}\;
    \text{FIR RRC} \;\xrightarrow{\;}\;
    \text{TX}_{I,Q}
    \label{eq:tx_chain}
\end{equation}

Los m\'{o}dulos RTL que materializan esta cadena, implementados en
SystemVerilog, son los siguientes:

\begin{enumerate}[label=\textbf{\arabic*.}]
    \item \texttt{bit\_gen} --- Generador de bits pseudo-aleatorios (PRBS-23).
    \item \texttt{qam16\_mapper} --- Mapeador 16-QAM con codificaci\'{o}n Gray.
    \item \texttt{rrc\_filter} --- Filtro FIR Root-Raised Cosine (5 taps).
    \item \texttt{tx\_top} --- Integraci\'{o}n del subsistema completo,
          incluyendo el sobremuestreo por inserci\'{o}n de ceros.
\end{enumerate}

Todos los m\'{o}dulos comparten par\'{a}metros, tipos y constantes definidos
en el paquete centralizado \texttt{gdsp\_pkg}, lo que garantiza la
coherencia de la aritm\'{e}tica de punto fijo a lo largo de toda la
jerarqu\'{\i}a de dise\~{n}o.

La plataforma objetivo impone restricciones estrictas de recursos
(\num{8640}~LUT4, 20~bloques DSP MULT9, \SI{27}{\mega\hertz} de reloj
del sistema) que han condicionado cada decisi\'{o}n arquitect\'{o}nica,
como se detalla en las secciones siguientes.


% ============================================================================
\section{Aritm\'{e}tica de Punto Fijo: Formato Q1.11}
\label{sec:fixed_point}
% ============================================================================

% ----------------------------------------------------------------------------
\subsection{Justificaci\'{o}n del formato de 12 bits firmados}
\label{subsec:fp_justificacion}
% ----------------------------------------------------------------------------

La elecci\'{o}n de la representaci\'{o}n num\'{e}rica es una decisi\'{o}n
cr\'{\i}tica en todo dise\~{n}o DSP embebido, ya que afecta directamente
a la precisi\'{o}n de la se\~{n}al, al consumo de recursos y al rendimiento
del camino cr\'{\i}tico. En el proyecto G-DSP Engine se adopta el formato
\textbf{Q1.11} en complemento a dos:

\begin{itemize}
    \item \textbf{1 bit de signo} (impl\'{\i}cito como MSB del complemento a dos).
    \item \textbf{0 bits enteros} (excluyendo el signo).
    \item \textbf{11 bits fraccionarios}.
    \item \textbf{Anchura total}: $N = 1 + 0 + 11 = 12$ bits.
\end{itemize}

Se emplea la convenci\'{o}n de Texas Instruments, donde la notaci\'{o}n
$Q_{m.n}$ indica $m$ bits enteros (incluyendo el bit de signo) y $n$ bits
fraccionarios, con una anchura total de $m + n$ bits. As\'{\i}, Q1.11
representa:

\begin{equation}
    x = -b_{11} \cdot 2^{0} + \sum_{k=0}^{10} b_k \cdot 2^{k-11}
    \label{eq:q1_11_def}
\end{equation}

donde $b_{11}$ es el MSB (bit de signo) y $b_0$ es el LSB.

El rango representable es:

\begin{equation}
    x \in \left[-1{,}0\;;\; +1{,}0 - 2^{-11}\right]
    = \left[-1{,}0\;;\; +0{,}99951\right]
    \label{eq:rango_q1_11}
\end{equation}

con una resoluci\'{o}n (paso de cuantificaci\'{o}n) de:

\begin{equation}
    \Delta = 2^{-11} = \frac{1}{2048} \approx \num{4,88e-4}
    \label{eq:resolucion}
\end{equation}

\paragraph{Implicaciones para la modulaci\'{o}n.}
Los niveles de la constelaci\'{o}n 16-QAM normalizados a potencia unitaria
tienen valor m\'{a}ximo $|3/\sqrt{10}| \approx 0{,}9487$, que cae
holgadamente dentro del rango $[-1, +0{,}9995]$, dejando un margen del
$5{,}4\,\%$ para acomodar el \textit{overshoot} t\'{\i}pico de la
conformaci\'{o}n espectral RRC (que depende del factor de roll-off
$\alpha$). Para $\alpha = 0{,}25$, el \textit{overshoot} pico-a-pico
del filtro RRC es inferior al $4\,\%$ del nivel de pico de la se\~{n}al,
por lo que el formato Q1.11 resulta suficiente sin riesgo de saturaci\'{o}n.

\paragraph{Adaptaci\'{o}n a los recursos DSP.}
Los bloques MULT9 del GW1NR-9 operan nativamente en modo $9 \times 9$ bits
firmados o pueden acoplarse en pares para formar multiplicadores de
$18 \times 18$ bits (modo MULT18). Una muestra de 12~bits y un coeficiente
de 12~bits requieren como m\'{\i}nimo el modo MULT18, lo que resulta
\'{o}ptimo: anchuras menores desperdiciar\'{\i}an capacidad DSP y anchuras
mayores (por ejemplo, 16~bits) requerir\'{\i}an acoplar cuatro MULT9,
duplicando el consumo de slices DSP.

% ----------------------------------------------------------------------------
\subsection{Relaci\'{o}n se\~{n}al a ruido de cuantificaci\'{o}n (SQNR)}
\label{subsec:sqnr}
% ----------------------------------------------------------------------------

Para un cuantificador uniforme de $B$ bits fraccionarios con se\~{n}al de
entrada sinusoidal a fondo de escala, la relaci\'{o}n se\~{n}al a ruido de
cuantificaci\'{o}n viene dada por la f\'{o}rmula cl\'{a}sica:

\begin{equation}
    \text{SQNR} \approx 6{,}02 \cdot B + 1{,}76 \;\text{dB}
    \label{eq:sqnr}
\end{equation}

Sustituyendo $B = 11$ (bits fraccionarios del formato Q1.11):

\begin{equation}
    \text{SQNR} \approx 6{,}02 \times 11 + 1{,}76
    = 66{,}22 + 1{,}76
    = \boxed{67{,}98\;\text{dB}}
    \label{eq:sqnr_val}
\end{equation}

Este valor de ${\sim}\SI{68}{\decibel}$ es ampliamente superior al
requisito t\'{\i}pico de un sistema 16-QAM, que opera con
$E_b/N_0 \sim \SI{10}{\decibel}$ en el umbral de BER objetivo
($10^{-5}$). El ruido de cuantificaci\'{o}n queda, por tanto, m\'{a}s de
\SI{55}{\decibel} por debajo del ruido t\'{e}rmico del canal, siendo
despreciable a efectos pr\'{a}cticos.

% ----------------------------------------------------------------------------
\subsection{Crecimiento de bits en la cadena de procesado}
\label{subsec:bit_growth}
% ----------------------------------------------------------------------------

En toda cadena DSP, las operaciones aritm\'{e}ticas incrementan
progresivamente la anchura de palabra. La Tabla~\ref{tab:bit_growth}
resume la evoluci\'{o}n a lo largo del subsistema TX.

\begin{table}[H]
    \centering
    \caption{Crecimiento de bits a lo largo de la cadena TX.}
    \label{tab:bit_growth}
    \begin{tabular}{@{} l c c l @{}}
        \toprule
        \textbf{Etapa} & \textbf{Formato} & \textbf{Bits} & \textbf{Justificaci\'{o}n} \\
        \midrule
        Muestra de entrada      & Q1.11 & 12 & Se\~{n}al I/Q original \\
        Coeficiente FIR         & Q1.11 & 12 & Misma anchura que datos \\
        Producto (dato $\times$ coef.) & Q2.22 & 24 & $12 + 12 = 24$ \\
        Acumulador (27 bits)    & Q5.22 & 27 & $24 + \lceil\log_2(5)\rceil = 27$ \\
        Salida truncada         & Q1.11 & 12 & Redondeo convergente \\
        \bottomrule
    \end{tabular}
\end{table}

Los 3~bits adicionales del acumulador ($\lceil\log_2(5)\rceil = 3$)
previenen cualquier desbordamiento durante la acumulaci\'{o}n de hasta
5~t\'{e}rminos, garantizando que la precisi\'{o}n num\'{e}rica del filtro
est\'{e} limitada \'{u}nicamente por la truncaci\'{o}n final, y no por
errores de desbordamiento intermedios.


% ============================================================================
\section{Generaci\'{o}n de Bits: PRBS-23 (LFSR)}
\label{sec:bit_gen}
% ============================================================================

% ----------------------------------------------------------------------------
\subsection{Polinomio generador seg\'{u}n ITU-T O.151}
\label{subsec:lfsr_poly}
% ----------------------------------------------------------------------------

Para la generaci\'{o}n de la secuencia de datos de prueba se emplea un
\textit{Linear Feedback Shift Register} (LFSR) de longitud m\'{a}xima,
conforme a la recomendaci\'{o}n ITU-T O.151. El polinomio caracter\'{\i}stico
es:

\begin{equation}
    p(x) = x^{23} + x^{18} + 1
    \label{eq:lfsr_poly}
\end{equation}

Este polinomio es primitivo sobre $\text{GF}(2)$, lo que garantiza que la
secuencia generada tiene per\'{\i}odo m\'{a}ximo:

\begin{equation}
    L = 2^{23} - 1 = \num{8388607} \;\text{bits}
    \label{eq:lfsr_period}
\end{equation}

Para la modulaci\'{o}n 16-QAM, con 4 bits por s\'{\i}mbolo, esto equivale
a $L/4 = \num{2097151}$ s\'{\i}mbolos antes de que la secuencia se repita.
A una tasa de s\'{\i}mbolo de $f_{\text{sym}} = 27/4 = \SI{6,75}{\mega\hertz}$,
el per\'{\i}odo de repetici\'{o}n es:

\begin{equation}
    T_{\text{rep}} = \frac{2^{23}-1}{4 \cdot f_{\text{sym}}}
    = \frac{\num{8388607}}{4 \times \num{6,75e6}}
    \approx \SI{0,311}{\second}
    \label{eq:period_time}
\end{equation}

Este per\'{\i}odo es suficientemente largo para que las propiedades
estad\'{\i}sticas de la secuencia sean indistinguibles de un proceso
verdaderamente aleatorio en el contexto de las pruebas de verificaci\'{o}n.

La elecci\'{o}n de una secuencia PRBS normalizada (frente a un generador
aleatorio arbitrario) se justifica por dos razones:

\begin{enumerate}
    \item \textbf{Repetibilidad}: permite la verificaci\'{o}n \textit{bit-true}
          contra el modelo de referencia Python (\textit{Golden Model}).
    \item \textbf{Propiedades espectrales}: la PRBS-23 presenta un espectro
          plano en la banda de inter\'{e}s, lo que la convierte en una
          excitaci\'{o}n adecuada para la medida de respuesta del sistema.
\end{enumerate}

% ----------------------------------------------------------------------------
\subsection{Desenrollado del LFSR para entrega paralela}
\label{subsec:lfsr_unroll}
% ----------------------------------------------------------------------------

El mapeador 16-QAM requiere 4~bits por s\'{\i}mbolo en cada ciclo de
reloj en el que se genera un nuevo s\'{\i}mbolo (cada $\text{SPS} = 4$
ciclos de reloj del sistema). Un LFSR convencional produce un \'{u}nico
bit por ciclo, lo que obligar\'{\i}a a acumular bits durante 4 ciclos
consecutivos e introducir\'{\i}a latencia y complejidad de control
innecesarias.

La soluci\'{o}n adoptada es el \textbf{desenrollado combinacional}
(\textit{loop unrolling}) del LFSR: se calcula la funci\'{o}n de transici\'{o}n
de estado 4 veces en cascada dentro de un mismo ciclo de reloj, obteniendo
los 4 bits de salida simult\'{a}neamente. La funci\'{o}n de avance unitario
se define como:

\begin{equation}
    \text{fb} = s[22] \oplus s[17], \qquad
    s' = \{s[21:0],\; \text{fb}\}
    \label{eq:lfsr_step}
\end{equation}

donde $s[k]$ denota el bit $k$-\'{e}simo del estado del LFSR (indexaci\'{o}n
desde cero) y $\oplus$ es la operaci\'{o}n XOR. El desenrollado produce
una cadena de estados intermedios:

\begin{equation}
    s^{(0)} = s_{\text{actual}}, \quad
    s^{(i+1)} = f_{\text{step}}(s^{(i)}), \quad
    i \in \{0,1,2,3\}
    \label{eq:unroll_chain}
\end{equation}

El bit de salida de cada paso es el MSB del estado intermedio
$s^{(i)}[22]$, y los 4~bits se recogen en paralelo:

\begin{equation}
    \texttt{bits\_out}[3-i] = s^{(i)}[22], \qquad i \in \{0,1,2,3\}
    \label{eq:bits_parallel}
\end{equation}

El nuevo estado del LFSR se actualiza al estado final $s^{(4)}$ en el
flanco de reloj, y todo el c\'{a}lculo combinacional intermedio se resuelve
dentro de un \'{u}nico per\'{\i}odo de reloj. Dado que cada paso del
desenrollado implica \'{u}nicamente una puerta XOR y un desplazamiento,
el camino cr\'{\i}tico total del desenrollado de 4 etapas es de apenas
4 niveles de XOR, lo que resulta trivial para una frecuencia de operaci\'{o}n
de \SI{27}{\mega\hertz} ($T_{\text{clk}} \approx \SI{37}{\nano\second}$).

La semilla inicial se fija en el valor \texttt{0x7FFFFF} (\textit{all-ones}),
evitando el estado de bloqueo $s = 0$ que es inherente a todo LFSR est\'{a}ndar.
La carga de la semilla se efect\'{u}a mediante la se\~{n}al de reset
as\'{\i}ncrono \texttt{rst\_n}.


% ============================================================================
\section{Mapeador 16-QAM con Codificaci\'{o}n Gray}
\label{sec:qam_mapper}
% ============================================================================

% ----------------------------------------------------------------------------
\subsection{Codificaci\'{o}n Gray por eje}
\label{subsec:gray_coding}
% ----------------------------------------------------------------------------

La modulaci\'{o}n 16-QAM asigna bloques de $\log_2(16) = 4$ bits a puntos
de una constelaci\'{o}n bidimensional. El mapeador implementado separa los
4~bits en dos pares independientes: los bits~$[3:2]$ controlan el eje
en fase (I) y los bits~$[1:0]$ controlan el eje en cuadratura (Q).

Para cada eje se aplica una \textbf{codificaci\'{o}n Gray}, en la que
s\'{\i}mbolos adyacentes en la constelaci\'{o}n difieren en exactamente
un bit. Esto minimiza la probabilidad de error de bit (BER) para una
probabilidad de error de s\'{\i}mbolo (SER) dada, ya que la mayor\'{\i}a
de errores de detecci\'{o}n se producen entre s\'{\i}mbolos vecinos.

La correspondencia entre los 2~bits de cada eje y el nivel de la
constelaci\'{o}n se recoge en la Tabla~\ref{tab:gray_lut}.

\begin{table}[H]
    \centering
    \caption{Tabla de correspondencia Gray para un eje de la constelaci\'{o}n 16-QAM.}
    \label{tab:gray_lut}
    \begin{tabular}{@{} c c c c @{}}
        \toprule
        \textbf{Bits} & \textbf{Nivel} & \textbf{Valor normalizado} & \textbf{Q1.11 (decimal)} \\
        \midrule
        \texttt{00} & $-3$ & $-3/\sqrt{10} \approx -0{,}9487$ & $-1943$ \\
        \texttt{01} & $-1$ & $-1/\sqrt{10} \approx -0{,}3162$ & $-648$  \\
        \texttt{11} & $+1$ & $+1/\sqrt{10} \approx +0{,}3162$ & $+648$  \\
        \texttt{10} & $+3$ & $+3/\sqrt{10} \approx +0{,}9487$ & $+1943$ \\
        \bottomrule
    \end{tabular}
\end{table}

\noindent
N\'{o}tese que la transici\'{o}n entre niveles adyacentes ($-3 \to -1$,
$-1 \to +1$, $+1 \to +3$) cambia exactamente un bit en cada caso:
$\texttt{00} \to \texttt{01} \to \texttt{11} \to \texttt{10}$, como
exige el c\'{o}digo Gray.

% ----------------------------------------------------------------------------
\subsection{Normalizaci\'{o}n de potencia unitaria}
\label{subsec:qam_normalization}
% ----------------------------------------------------------------------------

En un sistema de comunicaciones digital, es pr\'{a}ctica est\'{a}ndar
normalizar la constelaci\'{o}n para que su potencia media sea la unidad.
Para una constelaci\'{o}n 16-QAM rectangular con niveles $\{-3, -1, +1, +3\}$
equip-robables en cada eje, la potencia media por eje es:

\begin{equation}
    P_{\text{eje}} = \frac{1}{4}\left((-3)^2 + (-1)^2 + (+1)^2 + (+3)^2\right)
    = \frac{9 + 1 + 1 + 9}{4}
    = 5
    \label{eq:potencia_eje}
\end{equation}

La potencia total de la constelaci\'{o}n (suma de ambos ejes) es:

\begin{equation}
    P_{\text{total}} = 2 \times P_{\text{eje}} = 10
    \label{eq:potencia_total}
\end{equation}

Para obtener potencia media unitaria ($E[|s|^2] = 1$), los niveles
se dividen por $\sqrt{P_{\text{total}}} = \sqrt{10}$:

\begin{equation}
    d_k = \frac{a_k}{\sqrt{10}}, \qquad a_k \in \{-3, -1, +1, +3\}
    \label{eq:normalizacion}
\end{equation}

Los valores normalizados y su representaci\'{o}n en punto fijo Q1.11
($\text{valor entero} = \text{round}(d_k \times 2^{11})$) se resumen
en la Tabla~\ref{tab:qam_levels}.

\begin{table}[H]
    \centering
    \caption{Niveles 16-QAM normalizados y su cuantificaci\'{o}n en Q1.11.}
    \label{tab:qam_levels}
    \begin{tabular}{@{} c r r r r @{}}
        \toprule
        \textbf{Nivel} $a_k$ &
        \textbf{Flotante} $d_k$ &
        \textbf{Ideal} $d_k \!\times\! 2^{11}$ &
        \textbf{Entero Q1.11} &
        \textbf{Hex (12 bits)} \\
        \midrule
        $-3$ & $-0{,}948683$ & $-1942{,}90$ & $-1943$ & \texttt{0x869} \\
        $-1$ & $-0{,}316228$ & $-647{,}63$  & $-648$  & \texttt{0xD78} \\
        $+1$ & $+0{,}316228$ & $+647{,}63$  & $+648$  & \texttt{0x288} \\
        $+3$ & $+0{,}948683$ & $+1942{,}90$ & $+1943$ & \texttt{0x797} \\
        \bottomrule
    \end{tabular}
\end{table}

El error de cuantificaci\'{o}n m\'{a}ximo es:

\begin{equation}
    \epsilon_{\max} = \frac{|1943 - 1942{,}90|}{2^{11}}
    = \frac{0{,}10}{2048}
    \approx \num{4,88e-5}
    \label{eq:quant_error}
\end{equation}

\noindent
lo que supone un error relativo del $0{,}005\,\%$ respecto al nivel
m\'{a}ximo de la constelaci\'{o}n, completamente despreciable.

La implementaci\'{o}n RTL utiliza una funci\'{o}n combinacional
\texttt{gray\_lut()} que recibe los 2~bits de un eje y devuelve la
constante Q1.11 correspondiente, definida como par\'{a}metro en el paquete
\texttt{gdsp\_pkg}:

\begin{lstlisting}[caption={LUT Gray para un eje de la constelaci\'{o}n.},
                   label={lst:gray_lut}]
function automatic sample_t gray_lut(input logic [1:0] bits);
    case (bits)
        2'b00:   gray_lut = QAM_NEG3;   // -1943
        2'b01:   gray_lut = QAM_NEG1;   //  -648
        2'b11:   gray_lut = QAM_POS1;   //  +648
        2'b10:   gray_lut = QAM_POS3;   // +1943
        default: gray_lut = '0;
    endcase
endfunction
\end{lstlisting}

La salida del mapeador se registra (1 ciclo de latencia) para aislar
el \textit{fanout} de la LUT del camino cr\'{\i}tico del filtro RRC aguas
abajo.


% ============================================================================
\section{Filtro Root-Raised Cosine (RRC)}
\label{sec:rrc_filter}
% ============================================================================

% ----------------------------------------------------------------------------
\subsection{Par\'{a}metros del filtro}
\label{subsec:rrc_params}
% ----------------------------------------------------------------------------

El filtro de conformaci\'{o}n espectral utilizado es un \textit{Root-Raised
Cosine} (RRC), cuya respuesta al impulso en tiempo continuo viene dada
por:

\begin{equation}
    h_{\text{RRC}}(t) =
    \begin{cases}
        \dfrac{1}{T_s}\left(1 + \alpha\left(\dfrac{4}{\pi} - 1\right)\right),
        & t = 0 \\[1.2em]
        \dfrac{\alpha}{T_s\sqrt{2}}\left[\left(1+\dfrac{2}{\pi}\right)
        \sin\!\left(\dfrac{\pi}{4\alpha}\right)
        + \left(1-\dfrac{2}{\pi}\right)
        \cos\!\left(\dfrac{\pi}{4\alpha}\right)\right],
        & t = \pm\dfrac{T_s}{4\alpha} \\[1.2em]
        \dfrac{1}{T_s}\,\dfrac{\sin\!\bigl[\pi\frac{t}{T_s}(1-\alpha)\bigr]
        + 4\alpha\frac{t}{T_s}\cos\!\bigl[\pi\frac{t}{T_s}(1+\alpha)\bigr]}
        {\pi\frac{t}{T_s}\bigl[1-\bigl(4\alpha\frac{t}{T_s}\bigr)^2\bigr]},
        & \text{resto}
    \end{cases}
    \label{eq:rrc_impulse}
\end{equation}

Los par\'{a}metros de dise\~{n}o seleccionados son:

\begin{table}[H]
    \centering
    \caption{Par\'{a}metros de dise\~{n}o del filtro RRC.}
    \label{tab:rrc_params}
    \begin{tabular}{@{} l c l @{}}
        \toprule
        \textbf{Par\'{a}metro} & \textbf{Valor} & \textbf{Justificaci\'{o}n} \\
        \midrule
        N\'{u}mero de taps        & 5      & Limitado por recursos DSP (100\% uso) \\
        Factor de roll-off $\alpha$ & 0,25 & Est\'{a}ndar DVB-S2, exceso de
                                              ancho de banda moderado \\
        Muestras por s\'{\i}mbolo (SPS) & 4 & Cumple Nyquist ($f_s > 2f_{\text{sym}}$) \\
        Coef. pico (tap central)  & $+922$ & $0{,}450195$ en Q1.11
                                              ($< 0{,}5$, sin saturaci\'{o}n) \\
        \bottomrule
    \end{tabular}
\end{table}

\textbf{Nota sobre limitaci\'{o}n de recursos:} El dise\~{n}o original contemplaba
33~taps para una respuesta de pulso \'{o}ptima con atenuaci\'{o}n de
l\'{o}bulos laterales $> 40$~dB. Sin embargo, la FPGA GW1NR-9 utiliza el
100\% de sus 10 bloques DSP MULT18 con la arquitectura actual (filtros I/Q TX,
filtros I/Q RX matched, interpolador Gardner y derotador Costas). Se opt\'{o}
por reducir a 5~taps, aceptando aproximadamente 20\% de ISI residual a cambio
de mantener toda la funcionalidad del m\'{o}dem en el dispositivo objetivo.
El ISI se gestiona mediante una dead zone en los lazos de sincronismo.

La simetr\'{\i}a de fase lineal del filtro ($h[n] = h[N-1-n]$) se verifica
examinando los coeficientes generados:

\begin{equation}
    h[0] = h[4] = +536, \quad
    h[1] = h[3] = +814, \quad
    h[2] = +922 \;\text{(tap central)}
    h[16] = +922 \;\text{(tap central)}
    \label{eq:symmetry}
\end{equation}

Esta simetr\'{\i}a es una propiedad fundamental que garantiza un retardo
de grupo constante de $(N-1)/2 = 16$ muestras para todas las frecuencias
dentro de la banda de paso.

% ----------------------------------------------------------------------------
\subsection{Decisi\'{o}n de arquitectura: forma transpuesta}
\label{subsec:transposed_form}
% ----------------------------------------------------------------------------

Para la implementaci\'{o}n hardware del filtro FIR se han evaluado tres
arquitecturas cl\'{a}sicas:

\paragraph{1. Forma directa.}
En la forma directa, las muestras de entrada se almacenan en una l\'{\i}nea
de retardo y se multiplican por los coeficientes. Los productos se
acumulan mediante una cadena de sumadores. El problema principal es que
el \textbf{camino cr\'{\i}tico} crece linealmente con el n\'{u}mero de
taps, ya que la acumulaci\'{o}n es secuencial:

\begin{equation}
    t_{\text{cr\'{\i}tico}}^{\text{directa}} = t_{\text{mult}}
    + (N-1) \cdot t_{\text{add}}
    \label{eq:cp_direct}
\end{equation}

Para $N = 33$, esto resulta en 32 etapas de sumador en serie, haciendo
muy dif\'{\i}cil cumplir el requisito de \textit{timing} a
\SI{27}{\mega\hertz}.

\paragraph{2. Arquitectura plegada (\textit{folded}).}
La alternativa plegada reutiliza un \'{u}nico multiplicador en m\'{u}ltiples
ciclos de reloj. Para 33~taps se necesitar\'{\i}an 33 multiplicaciones por
muestra, lo que exige un reloj interno de:

\begin{equation}
    f_{\text{interno}} = N \times f_s = 33 \times \SI{27}{\mega\hertz}
    = \SI{891}{\mega\hertz}
    \label{eq:folded_clk}
\end{equation}

Esta frecuencia es inalcanzable para la familia GW1NR-9 (cuyo l\'{\i}mite
pr\'{a}ctico se sit\'{u}a en torno a \SI{200}{\mega\hertz} para l\'{o}gica
general), por lo que se descarta esta opci\'{o}n.

\paragraph{3. Forma transpuesta (\textbf{selecci\'{o}n final}).}
En la forma transpuesta, la estructura se invierte: cada etapa consiste
en un multiplicador seguido de un sumador y un registro. El camino
cr\'{\i}tico se reduce a \textbf{una \'{u}nica multiplicaci\'{o}n m\'{a}s
una suma}:

\begin{equation}
    t_{\text{cr\'{\i}tico}}^{\text{transpuesta}} = t_{\text{mult}}
    + t_{\text{add}}
    \label{eq:cp_transposed}
\end{equation}

\noindent
independientemente del n\'{u}mero de taps. Esta caracter\'{\i}stica la
convierte en la arquitectura id\'{o}nea para implementaciones en FPGA con
restricciones de frecuencia.

Adem\'{a}s, la forma transpuesta se adapta naturalmente a los bloques DSP
de Gowin: cada etapa mapea directamente sobre la primitiva
\texttt{pREG + MULT9}, donde el registro post-multiplicador
(\texttt{pREG}) absorbe la registraci\'{o}n de la etapa del \textit{pipeline}
sin consumir LUTs adicionales.

La ecuaci\'{o}n de recurrencia de la forma transpuesta para un filtro de
$N$ taps es:

\begin{equation}
    \begin{cases}
        p_k[n]   = x[n] \cdot h[k], & 0 \leq k \leq N-1 \\
        r_{N-1}[n] = p_{N-1}[n] & \text{(\'{u}ltima etapa)} \\
        r_k[n]   = p_k[n] + r_{k+1}[n-1], & 0 \leq k \leq N-2 \\
        y[n]     = r_0[n-1] & \text{(salida)}
    \end{cases}
    \label{eq:transposed_eqs}
\end{equation}

\begin{figure}[H]
    \centering
    \begin{tikzpicture}[
        mult/.style={circle, draw, minimum size=8mm, inner sep=0pt,
                     font=\footnotesize},
        add/.style={circle, draw, minimum size=6mm, inner sep=0pt,
                    font=\scriptsize},
        reg/.style={rectangle, draw, minimum width=8mm, minimum height=6mm,
                    font=\footnotesize},
        >=Stealth,
        node distance=18mm and 15mm
    ]
        % Input
        \node (xn) {$x[n]$};

        % Stage 0
        \node[mult, right=12mm of xn] (m0) {$\times$};
        \node[below=3mm of m0, font=\scriptsize] {$h[0]$};
        \node[add, right=of m0] (a0) {$+$};
        \node[reg, right=of a0] (r0) {$z^{-1}$};

        % Stage 1
        \node[mult, right=25mm of r0] (m1) {$\times$};
        \node[below=3mm of m1, font=\scriptsize] {$h[1]$};
        \node[add, right=of m1] (a1) {$+$};
        \node[reg, right=of a1] (r1) {$z^{-1}$};

        % Dots
        \node[right=15mm of r1, font=\Large] (dots) {$\cdots$};

        % Stage N-1
        \node[mult, right=15mm of dots] (mN) {$\times$};
        \node[below=3mm of mN, font=\scriptsize] {$h[N\!-\!1]$};
        \node[reg, right=15mm of mN] (rN) {$z^{-1}$};

        % Output
        \node[right=12mm of r0] (yn) {};
        \node[above=2mm of r0, font=\footnotesize] {$y[n]$};

        % Connections
        \draw[->] (xn) -- (m0);
        \draw[->] (m0) -- (a0);
        \draw[->] (a0) -- (r0);
        \draw[->] (r0.east) -- ++(5mm,0) node[right, font=\footnotesize]{$y[n]$};

        % x[n] fan-out
        \draw[->] (xn) |- ([yshift=12mm]m0.north) -| (m1);
        \draw[->] ([yshift=12mm]m1.north) -| (mN);

        % Backward connections (from register to adder)
        \draw[->] (r1.west) -- (a0.east |- r1.west) -- (a0.east);
        \draw[->] (rN.west) -- (a1.east |- rN.west) -- (a1.east);

        % Stage N-1 direct
        \draw[->] (mN) -- (rN);
    \end{tikzpicture}
    \caption{Arquitectura FIR en forma transpuesta. El camino cr\'{\i}tico
             (l\'{\i}nea roja conceptual) es una \'{u}nica multiplicaci\'{o}n
             m\'{a}s una suma, independientemente de $N$.}
    \label{fig:transposed_fir}
\end{figure}

\paragraph{Presupuesto de recursos DSP.}
La FPGA GW1NR-9 dispone de 20 bloques MULT9, que operan como 10~bloques
MULT18 en modo acoplado. La multiplicaci\'{o}n de 12~bits firmados requiere
MULT18, por lo que se dispone de un m\'{a}ximo de 10~multiplicadores
hardware. Con la implementaci\'{o}n actual de 5~taps por filtro, el sistema
completo (TX RRC I/Q, RX RRC matched I/Q, interpolador Gardner y derotador
Costas) utiliza el 100\% de los bloques DSP. Esta limitaci\'{o}n conden\'{o}
la reducci\'{o}n de 33~taps a 5~taps respecto al dise\~{n}o te\'{o}rico inicial.

% ----------------------------------------------------------------------------
\subsection{An\'{a}lisis de precisi\'{o}n: redondeo convergente}
\label{subsec:convergent_rounding}
% ----------------------------------------------------------------------------

La operaci\'{o}n m\'{a}s delicada del filtro RRC es la
\textbf{truncaci\'{o}n} del acumulador de 27~bits al formato de salida
Q1.11 (12~bits). Una truncaci\'{o}n simple (descartar los bits menos
significativos) introduce un sesgo sistem\'{a}tico negativo (\textit{bias}
de DC) de valor medio $-\Delta/2 = -2^{-12}$, que se acumula a lo largo
de la cadena de procesado y puede degradar el rendimiento del receptor.

Para eliminar este sesgo se emplea \textbf{redondeo convergente}
(\textit{Banker's rounding}, IEEE~754), que redondea al valor par
m\'{a}s cercano cuando la fracci\'{o}n descartada es exactamente $0{,}5$:

\begin{equation}
    y = \lfloor x \rfloor +
    \begin{cases}
        1, & \text{si } r > 0{,}5 \\
        1, & \text{si } r = 0{,}5 \;\text{y}\; \lfloor x \rfloor \;\text{es impar} \\
        0, & \text{en otro caso}
    \end{cases}
    \label{eq:convergent_round}
\end{equation}

donde $r$ es la parte fraccionaria descartada. En la implementaci\'{o}n
hardware, esta condici\'{o}n se codifica de forma eficiente mediante tres
se\~{n}ales derivadas del acumulador \texttt{acc\_val}:

\begin{table}[H]
    \centering
    \caption{Se\~{n}ales para el redondeo convergente.}
    \label{tab:round_signals}
    \begin{tabular}{@{} l c l @{}}
        \toprule
        \textbf{Se\~{n}al} & \textbf{Bits del acumulador} & \textbf{Significado} \\
        \midrule
        \texttt{round\_bit}  & \texttt{acc\_val[10]} & MSB de la parte descartada \\
        \texttt{sticky}      & \texttt{|acc\_val[9:0]} & OR de los bits restantes \\
        \texttt{lsb\_result} & \texttt{acc\_val[11]} & LSB del resultado truncado \\
        \bottomrule
    \end{tabular}
\end{table}

La decisi\'{o}n de redondeo al alza se codifica como:

\begin{equation}
    \texttt{round\_up} = \texttt{round\_bit} \;\wedge\;
    (\texttt{sticky} \;\vee\; \texttt{lsb\_result})
    \label{eq:round_up}
\end{equation}

Esta expresi\'{o}n garantiza que:

\begin{itemize}
    \item Si la fracci\'{o}n descartada es $> 0{,}5$ (\texttt{round\_bit}~=~1
          y \texttt{sticky}~=~1), se redondea al alza.
    \item Si la fracci\'{o}n es exactamente $0{,}5$ (\texttt{round\_bit}~=~1
          y \texttt{sticky}~=~0), se redondea al alza solo si el LSB del
          resultado es 1 (n\'{u}mero impar), forzando el resultado a ser par.
    \item Si la fracci\'{o}n es $< 0{,}5$ (\texttt{round\_bit}~=~0), se
          trunca (redondeo a la baja).
\end{itemize}

La salida truncada y redondeada se obtiene como:

\begin{equation}
    \texttt{dout\_trunc} = \texttt{acc\_val}[22:11] + \texttt{round\_up}
    \label{eq:trunc}
\end{equation}

donde los bits $[22:11]$ representan los 12~bits Q1.11 del producto
acumulado Q2.22. Adicionalmente, se incorpora un circuito de
\textbf{saturaci\'{o}n} que protege contra desbordamientos inesperados,
limitando la salida a los extremos representables:

\begin{equation}
    y = \begin{cases}
        +2047  & \text{si desbordamiento positivo (SAT\_POS)} \\
        -2048  & \text{si desbordamiento negativo (SAT\_NEG)} \\
        \texttt{dout\_trunc} & \text{en caso normal}
    \end{cases}
    \label{eq:saturation}
\end{equation}

Con coeficientes correctamente normalizados ($\sum|h[n]| < 1{,}0$), la
saturaci\'{o}n nunca deber\'{\i}a activarse, pero su inclusi\'{o}n es una
pr\'{a}ctica de dise\~{n}o \textit{senior} que proporciona robustez frente
a errores de configuraci\'{o}n o condiciones de prueba an\'{o}malas.


% ============================================================================
\section{Integraci\'{o}n del Subsistema TX y \textit{Timing}}
\label{sec:tx_integration}
% ============================================================================

% ----------------------------------------------------------------------------
\subsection{Sobremuestreo por inserci\'{o}n de ceros}
\label{subsec:upsampling}
% ----------------------------------------------------------------------------

El filtro RRC opera a la tasa de muestreo del sistema
($f_s = \SI{27}{\mega\hertz}$), mientras que los s\'{\i}mbolos QAM se
generan a la tasa de s\'{\i}mbolo
($f_{\text{sym}} = f_s / \text{SPS} = \SI{6,75}{\mega\hertz}$). La
conversi\'{o}n entre ambas tasas se realiza mediante un
\textbf{sobremuestreo por inserci\'{o}n de ceros} con factor $L = 4$:

\begin{equation}
    x_{\text{up}}[n] =
    \begin{cases}
        s[n/L] & \text{si } n \bmod L = 0 \\
        0      & \text{en otro caso}
    \end{cases}
    \label{eq:upsample}
\end{equation}

donde $s[k]$ es la secuencia de s\'{\i}mbolos y $x_{\text{up}}[n]$ es la
secuencia sobremuestreada. Esta operaci\'{o}n crea un \textit{tren de
impulsos} que, al ser filtrado por el FIR RRC, produce la forma de pulso
RRC deseada.

En la implementaci\'{o}n RTL (\texttt{tx\_top.sv}), el sobremuestreador
utiliza una m\'{a}quina de estados con un contador \texttt{up\_cnt}:

\begin{enumerate}
    \item Cuando \texttt{sym\_valid} se activa, el valor del s\'{\i}mbolo
          ($I$, $Q$) se presenta a la salida y el contador se inicializa a~1.
    \item Durante los $\text{SPS} - 1 = 3$ ciclos siguientes, la salida
          se fuerza a cero (\textit{zero-stuffing}).
    \item El \texttt{up\_valid} permanece activo durante los 4~ciclos,
          alimentando al filtro RRC con una muestra v\'{a}lida en cada
          flanco de reloj.
\end{enumerate}

% ----------------------------------------------------------------------------
\subsection{Cadena de latencia del subsistema TX}
\label{subsec:tx_latency}
% ----------------------------------------------------------------------------

La latencia total del subsistema, desde la generaci\'{o}n de bits hasta la
salida del filtro, se compone de:

\begin{table}[H]
    \centering
    \caption{Desglose de latencia del subsistema TX.}
    \label{tab:latency}
    \begin{tabular}{@{} l c l @{}}
        \toprule
        \textbf{M\'{o}dulo} & \textbf{Latencia (ciclos)} & \textbf{Nota} \\
        \midrule
        \texttt{bit\_gen}       & 1 & Registro de salida \\
        \texttt{qam16\_mapper}  & 1 & Registro de salida \\
        \texttt{upsampler}      & 0 & Combinacional (integrado) \\
        \texttt{rrc\_filter}    & 1 & Pipeline transpuesto \\
        \midrule
        \textbf{Total pipeline} & \textbf{3} & Latencia m\'{\i}nima \\
        Retardo de grupo FIR    & 16 & $(N-1)/2 = 16$ muestras \\
        \midrule
        \textbf{Total efectivo} & \textbf{19} & $3 + 16$ muestras \\
        \bottomrule
    \end{tabular}
\end{table}

En t\'{e}rminos temporales, a \SI{27}{\mega\hertz}:

\begin{equation}
    t_{\text{latencia}} = \frac{19}{\SI{27}{\mega\hertz}}
    \approx \SI{0,704}{\micro\second}
    \label{eq:latency_time}
\end{equation}

% ----------------------------------------------------------------------------
\subsection{Generaci\'{o}n del \textit{strobe} de s\'{\i}mbolo}
\label{subsec:sym_strobe}
% ----------------------------------------------------------------------------

La sincronizaci\'{o}n entre la tasa de s\'{\i}mbolo y la tasa de muestreo
se gestiona mediante un contador m\'{o}dulo SPS implementado en
\texttt{tx\_top}:

\begin{lstlisting}[caption={Generador de \textit{strobe} de s\'{\i}mbolo.},
                   label={lst:sym_strobe}]
logic [$clog2(SPS)-1:0] sps_cnt;
logic sym_strobe;

always_ff @(posedge clk or negedge rst_n) begin
    if (!rst_n) begin
        sps_cnt    <= '0;
        sym_strobe <= 1'b0;
    end else if (en) begin
        if (sps_cnt == SPS - 1) begin
            sps_cnt    <= '0;
            sym_strobe <= 1'b1;
        end else begin
            sps_cnt    <= sps_cnt + 1'b1;
            sym_strobe <= 1'b0;
        end
    end
end
\end{lstlisting}

Este \textit{strobe} de un ciclo de duraci\'{o}n se propaga como se\~{n}al
de habilitaci\'{o}n (\texttt{en}) al generador de bits, garantizando que
se produce un nuevo s\'{\i}mbolo de 4~bits exactamente cada $\text{SPS} = 4$
ciclos de reloj.

% ----------------------------------------------------------------------------
\subsection{Uso de \texttt{gdsp\_pkg} para la mantenibilidad}
\label{subsec:gdsp_pkg}
% ----------------------------------------------------------------------------

Todas las constantes, anchuras de bus, tipos y par\'{a}metros del sistema
se centralizan en el paquete SystemVerilog \texttt{gdsp\_pkg.sv}. Esta
decisi\'{o}n arquitect\'{o}nica, com\'{u}n en dise\~{n}os industriales,
ofrece las siguientes ventajas:

\begin{itemize}
    \item \textbf{Coherencia}: un cambio en la anchura de palabra (por ejemplo,
          migrar de Q1.11 a Q1.15) se propaga autom\'{a}ticamente a todos los
          m\'{o}dulos mediante \texttt{import gdsp\_pkg::*}.
    \item \textbf{Seguridad de tipos}: el uso de \texttt{typedef}
          (\texttt{sample\_t}, \texttt{coeff\_t}, \texttt{product\_t},
          \texttt{accum\_t}) previene errores de anchura de bus en la
          instanciaci\'{o}n de m\'{o}dulos.
    \item \textbf{Documentaci\'{o}n impl\'{\i}cita}: los nombres de los
          par\'{a}metros (\texttt{FRAC\_BITS}, \texttt{NUM\_TAPS},
          \texttt{SPS}) hacen el c\'{o}digo auto-documentado.
    \item \textbf{Exploraci\'{o}n de dise\~{n}o}: facilita la exploraci\'{o}n
          param\'{e}trica (por ejemplo, variar el n\'{u}mero de taps o el
          factor SPS) sin modificar la l\'{o}gica de los m\'{o}dulos
          individuales.
\end{itemize}

La Tabla~\ref{tab:pkg_types} resume los tipos definidos en el paquete y su
correspondencia con la cadena de procesado.

\begin{table}[H]
    \centering
    \caption{Tipos definidos en \texttt{gdsp\_pkg} y su uso en la cadena TX.}
    \label{tab:pkg_types}
    \begin{tabular}{@{} l c l @{}}
        \toprule
        \textbf{Tipo} & \textbf{Anchura} & \textbf{Uso} \\
        \midrule
        \texttt{sample\_t}  & 12 bits & Muestras I/Q (entrada/salida del filtro) \\
        \texttt{coeff\_t}   & 12 bits & Coeficientes del filtro RRC \\
        \texttt{product\_t} & 24 bits & Producto dato $\times$ coeficiente \\
        \texttt{accum\_t}   & 27 bits & Acumulador del filtro FIR (5 taps) \\
        \bottomrule
    \end{tabular}
\end{table}


% ============================================================================
\section{S\'{\i}ntesis de decisiones de dise\~{n}o}
\label{sec:design_decisions}
% ============================================================================

A modo de resumen, la Tabla~\ref{tab:decisions} recopila las decisiones de
dise\~{n}o m\'{a}s relevantes del subsistema TX y la justificaci\'{o}n
t\'{e}cnica de cada una.

\begin{table}[H]
    \centering
    \caption{Resumen de decisiones de dise\~{n}o del subsistema TX.}
    \label{tab:decisions}
    \small
    \begin{tabular}{@{} p{3.2cm} p{3.0cm} p{7.5cm} @{}}
        \toprule
        \textbf{Aspecto} & \textbf{Decisi\'{o}n} & \textbf{Justificaci\'{o}n} \\
        \midrule
        Formato num\'{e}rico &
        Q1.11 (12 bits) &
        SQNR de \SI{68}{\decibel}; niveles QAM dentro de rango;
        compatible con MULT18 de Gowin. \\[0.5em]

        Generador de bits &
        PRBS-23 (ITU-T O.151) &
        Secuencia normalizada, per\'{\i}odo $>8$ millones de bits,
        repetibilidad para verificaci\'{o}n \textit{bit-true}. \\[0.5em]

        Entrega paralela &
        LFSR desenrollado $\times 4$ &
        Evita latencia de acumulaci\'{o}n; 4 niveles de XOR (trivial
        a \SI{27}{\mega\hertz}). \\[0.5em]

        Mapeador QAM &
        LUT Gray combinacional &
        Minimiza BER; 1 ciclo de latencia; constantes centralizadas
        en \texttt{gdsp\_pkg}. \\[0.5em]

        Normalizaci\'{o}n &
        $1/\sqrt{10}$ &
        Potencia media unitaria; valores enteros exactos en Q1.11
        ($\pm 1943$, $\pm 648$). \\[0.5em]

        Arquitectura FIR &
        Forma transpuesta &
        Camino cr\'{\i}tico m\'{\i}nimo ($t_\text{mult} + t_\text{add}$);
        mapeo directo a pREG+MULT9. \\[0.5em]

        Truncaci\'{o}n &
        Redondeo convergente &
        Elimina sesgo de DC; implementaci\'{o}n en 3 se\~{n}ales
        (round, sticky, LSB). \\[0.5em]

        Sobremuestreo &
        Inserci\'{o}n de ceros $\times 4$ &
        M\'{e}todo est\'{a}ndar; m\'{a}quina de estados de 2 bits;
        el FIR act\'{u}a como filtro anti-imagen. \\[0.5em]

        Parametrizaci\'{o}n &
        \texttt{gdsp\_pkg} centralizado &
        Coherencia, seguridad de tipos, exploraci\'{o}n param\'{e}trica
        sin modificar m\'{o}dulos. \\
        \bottomrule
    \end{tabular}
\end{table}


% ============================================================================
\section{Resultados de Simulaci\'{o}n}
\label{sec:sim_results}
% ============================================================================

Las siguientes figuras muestran los resultados de simulaci\'{o}n del
subsistema TX generados por el \textit{Golden Model} en Python.

\begin{figure}[H]
    \centering
    \includegraphics[width=0.7\textwidth]{../figures/constellation_tx.png}
    \caption{Constelaci\'{o}n 16-QAM a la salida del transmisor tras el
             filtrado RRC. Los 16 s\'{\i}mbolos se distribuyen en una
             cuadr\'{\i}cula 4$\times$4 con niveles normalizados.}
    \label{fig:const_tx}
\end{figure}

\begin{figure}[H]
    \centering
    \includegraphics[width=0.7\textwidth]{../figures/rrc_impulse.png}
    \caption{Respuesta al impulso del filtro RRC de 5 taps con
             $\alpha = 0{,}25$. La simetr\'{\i}a garantiza fase lineal.
             El tap central ($+922$ en Q1.11) es el de mayor magnitud.}
    \label{fig:rrc_impulse}
\end{figure}

\begin{figure}[H]
    \centering
    \includegraphics[width=0.7\textwidth]{../figures/spectrum_tx.png}
    \caption{Espectro de potencia de la se\~{n}al TX. El ancho de banda
             ocupado es $(1+\alpha) \cdot f_{sym} = 8{,}4$~MHz,
             exhibiendo la caracter\'{\i}stica ca\'{\i}da del RRC.}
    \label{fig:spectrum_tx}
\end{figure}

\begin{figure}[H]
    \centering
    \includegraphics[width=0.7\textwidth]{../figures/eye_diagram_I.png}
    \caption{Diagrama de ojo del canal I. La apertura vertical representa
             el margen de ruido disponible; la apertura horizontal indica
             el margen de \textit{timing}. El ISI residual del filtro de
             5~taps es visible como engrosamiento de las trazas.}
    \label{fig:eye_tx}
\end{figure}


% ============================================================================
% Fin del documento
% ============================================================================
\end{document}
