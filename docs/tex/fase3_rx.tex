% !TEX root = fase3_rx.tex
% ============================================================================
% G-DSP Engine --- Documentaci\'{o}n T\'{e}cnica: Fase 3
% Implementaci\'{o}n del Subsistema de Recepci\'{o}n (RX)
% ============================================================================
% Compilar:  pdflatex fase3_rx.tex  (x2 para referencias)
% ============================================================================

\documentclass[a4paper,11pt,spanish]{article}

% ---------------------------------------------------------------------------
% Paquetes
% ---------------------------------------------------------------------------
\usepackage[utf8]{inputenc}
\usepackage[T1]{fontenc}
\usepackage[spanish,es-tabla]{babel}
\usepackage{amsmath,amssymb,amsfonts}
\usepackage{booktabs}
\usepackage{array}
\usepackage{graphicx}
\usepackage{xcolor}
\usepackage{listings}
\usepackage{siunitx}
\usepackage{hyperref}
\usepackage{geometry}
\usepackage{caption}
\usepackage{subcaption}
\usepackage{enumitem}
\usepackage{float}
\usepackage{tikz}
\usetikzlibrary{arrows.meta, positioning, calc, shapes.geometric, decorations.pathreplacing}

\geometry{margin=2.5cm}

\sisetup{
    output-decimal-marker = {,},
    per-mode = symbol
}

% ---------------------------------------------------------------------------
% Configuraci\'{o}n de listings para SystemVerilog
% ---------------------------------------------------------------------------
\lstdefinelanguage{SystemVerilog}{
    morekeywords={module, endmodule, input, output, logic, signed, parameter,
                  int, always_ff, always_comb, posedge, negedge, if, else,
                  begin, end, for, case, endcase, function, automatic,
                  endfunction, localparam, assign, typedef, package,
                  endpackage, import, default, generate, endgenerate, wire},
    sensitive=true,
    morecomment=[l]{//},
    morecomment=[s]{/*}{*/},
    morestring=[b]",
}

\lstset{
    language=SystemVerilog,
    basicstyle=\ttfamily\footnotesize,
    keywordstyle=\bfseries\color{blue!70!black},
    commentstyle=\itshape\color{green!50!black},
    stringstyle=\color{red!60!black},
    numbers=left,
    numberstyle=\tiny\color{gray},
    numbersep=5pt,
    frame=single,
    breaklines=true,
    captionpos=b,
    tabsize=4,
    showstringspaces=false,
}

% ---------------------------------------------------------------------------
% Datos del documento
% ---------------------------------------------------------------------------
\title{%
    \textbf{G-DSP Engine} \\[0.5em]
    \Large Implementaci\'{o}n del Subsistema de Recepci\'{o}n \\[0.3em]
    \large Fase~3 --- Documentaci\'{o}n T\'{e}cnica del TFG
}
\author{G-DSP Team}
\date{Febrero 2026}

% ============================================================================
\begin{document}
% ============================================================================

\maketitle
\tableofcontents
\newpage

% ============================================================================
\section{Introducci\'{o}n al Subsistema de Recepci\'{o}n}
\label{sec:intro_rx}
% ============================================================================

El subsistema de recepci\'{o}n (RX) del procesador banda base G-DSP Engine
constituye la tercera fase funcional del m\'{o}dem 16-QAM, completando la
cadena de comunicaciones implementada sobre la FPGA Gowin
GW1NR-LV9QN88PC6/I5 (Tang Nano~9K).  Su funci\'{o}n es recuperar los
s\'{\i}mbolos transmitidos a partir de las muestras ruidosas entregadas por
el modelo de canal AWGN (Fase~2), resolviendo simult\'{a}neamente la
sincronizaci\'{o}n de tiempo de s\'{\i}mbolo y la recuperaci\'{o}n de fase
de portadora.

La cadena de recepci\'{o}n implementada sigue la arquitectura canónica
de un receptor QAM coherente:

\begin{equation}
    r_{I,Q}[n]
    \;\xrightarrow{\;}\;
    \text{FIR RRC (MF)}
    \;\xrightarrow{\;}\;
    \text{Gardner TED}
    \;\xrightarrow{\;1\,\text{sym}}\;
    \text{Costas DD-PLL}
    \;\xrightarrow{\;}\;
    \hat{I},\hat{Q}
    \label{eq:rx_chain}
\end{equation}

Los m\'{o}dulos RTL que materializan esta cadena son:

\begin{enumerate}[label=\textbf{\arabic*.}]
    \item \texttt{rrc\_filter} $\times 2$ --- Filtro acoplado (matched filter)
          para los canales~I y~Q.
    \item \texttt{gardner\_ted} --- Recuperaci\'{o}n de temporizaci\'{o}n
          con NCO e interpolaci\'{o}n lineal.
    \item \texttt{costas\_loop} --- Recuperaci\'{o}n de portadora mediante
          lazo de fase dirigido por decisiones (DD-PLL) para 16-QAM.
    \item \texttt{rx\_top} --- M\'{o}dulo de integraci\'{o}n que conecta
          los tres bloques anteriores.
\end{enumerate}


% ============================================================================
\section{Arquitectura del Receptor}
\label{sec:arch}
% ============================================================================

La Figura~\ref{fig:rx_block} muestra el diagrama de bloques del
subsistema de recepci\'{o}n.  La se\~{n}al de entrada proviene del
modelo de canal AWGN a tasa de muestreo ($f_s = \SI{27}{\mega\hertz}$,
4 muestras por s\'{\i}mbolo), y la salida son los s\'{\i}mbolos
demodulados a tasa de s\'{\i}mbolo ($f_\text{sym} = \SI{6,75}{\mega\hertz}$)
junto con un indicador de enganche (\texttt{demod\_lock}).

\begin{figure}[H]
\centering
\begin{tikzpicture}[
    block/.style={draw, minimum width=2.4cm, minimum height=1.4cm,
                  align=center, rounded corners=2pt, fill=blue!8},
    arr/.style={-{Stealth[length=6pt]}, thick},
    label/.style={font=\scriptsize, midway, above}
]
    % Bloques
    \node[block] (mf)  at (0,0)    {Filtro\\Acoplado\\(RRC)};
    \node[block] (ted) at (4.2,0)  {Gardner\\TED};
    \node[block] (cos) at (8.4,0)  {Costas\\DD-PLL};

    % Sub-bloques del Costas
    \node[font=\tiny, below=0pt of cos] {(NCO + LUT sin/cos)};

    % Entradas
    \draw[arr] (-2.3, 0.3) -- node[label] {$r_I[n]$} (mf.west |- 0,0.3);
    \draw[arr] (-2.3,-0.3) -- node[label,below] {$r_Q[n]$} (mf.west |- 0,-0.3);

    % Conexiones
    \draw[arr] (mf) -- node[label] {$\times$SPS} (ted);
    \draw[arr] (ted) -- node[label] {$1\times$sym} (cos);

    % Salidas
    \draw[arr] (cos.east |- 0,0.3) -- ++(1.8,0)
        node[right, font=\small] {$\hat{I},\hat{Q}$};
    \draw[arr] (cos.east |- 0,-0.3) -- ++(1.8,0)
        node[right, font=\small] {\texttt{lock}};

    % Retroalimentación Gardner
    \draw[arr, dashed, red!60!black]
        (ted.south) -- ++(0,-0.6) -| node[pos=0.25, below, font=\tiny]
        {ajuste NCO} (mf.south);

    % Anotaciones de tasa
    \node[font=\tiny, color=gray] at (0, -1.4) {tasa de muestra};
    \node[font=\tiny, color=gray] at (4.2, -1.0) {$\downarrow$4};
    \node[font=\tiny, color=gray] at (8.4, -1.4) {tasa de s\'{\i}mbolo};
\end{tikzpicture}
\caption{Diagrama de bloques del subsistema de recepci\'{o}n.  El filtro
acoplado opera a $f_s$; el Gardner TED decima a tasa de s\'{\i}mbolo;
el lazo de Costas corrige fase y frecuencia residual.}
\label{fig:rx_block}
\end{figure}


% ============================================================================
\section{Filtro Acoplado (Matched Filter)}
\label{sec:mf}
% ============================================================================

El filtro acoplado del receptor reutiliza el mismo m\'{o}dulo
\texttt{rrc\_filter} empleado en el transmisor (Fase~1), con coeficientes
RRC de longitud $L = 5$ taps y factor de \emph{roll-off} $\alpha = 0{,}25$.
Se instancian dos copias id\'{e}nticas, una para cada canal (I~y~Q).

\textbf{Nota:} La reducci\'{o}n de 33~taps a 5~taps, impuesta por la
limitaci\'{o}n de recursos DSP (100\% de utilizaci\'{o}n), introduce
aproximadamente un 20\% de ISI residual. Este efecto se mitiga mediante
las \emph{dead zones} implementadas en los lazos de sincronismo (Gardner y
Costas), que ignoran errores peque\~{n}os causados por la ISI.

\subsection{Justificaci\'{o}n}

En un sistema con conformaci\'{o}n de pulso \emph{split Nyquist}, la
cascada de los filtros RRC de transmisi\'{o}n y recepci\'{o}n produce una
respuesta total de coseno alzado (\emph{raised cosine}), que satisface el
primer criterio de Nyquist para interferencia intersimb\'{o}lica (ISI)
nula en los instantes \'{o}ptimos de muestreo:

\begin{equation}
    h_\text{RC}[n] = h_\text{RRC}[n] * h_\text{RRC}[n],
    \qquad
    h_\text{RC}[kT_s] = \begin{cases}
        1 & k = 0 \\
        0 & k \neq 0
    \end{cases}
    \label{eq:nyquist}
\end{equation}

Adem\'{a}s, el filtro acoplado maximiza la relaci\'{o}n se\~{n}al a ruido
(SNR) a la salida, propiedad fundamental para la detecci\'{o}n \'{o}ptima
en canales AWGN~\cite{haykin2014}.

\subsection{Compensaci\'{o}n de ganancia}

La cascada de dos filtros RRC discretos produce una ganancia de pico
$\|h_\text{RRC}\|^2 \approx 0{,}7095$ en lugar de la unidad te\'{o}rica
(debido a la discretizaci\'{o}n y truncamiento de los coeficientes a Q1.11).
Para restaurar los niveles de constelaci\'{o}n QAM a escala completa, el
m\'{o}dulo \texttt{rx\_top} aplica una etapa de compensaci\'{o}n de ganancia:

\begin{equation}
    y_\text{comp} = y_\text{MF} \times \frac{1443}{1024}
    \approx y_\text{MF} \times 1{,}4094
    \approx \frac{y_\text{MF}}{0{,}7095}
    \label{eq:gain_comp}
\end{equation}

Esta multiplicaci\'{o}n emplea 2~multiplicadores DSP adicionales (uno por
canal) a tasa de muestra.


% ============================================================================
\section{Recuperaci\'{o}n de Temporizaci\'{o}n --- Gardner TED}
\label{sec:gardner}
% ============================================================================

La recuperaci\'{o}n de temporizaci\'{o}n (\emph{timing recovery}) es
necesaria para identificar los instantes \'{o}ptimos de muestreo dentro
de cada per\'{\i}odo de s\'{\i}mbolo.  El m\'{o}dulo \texttt{gardner\_ted}
implementa el algoritmo de Gardner, un detector de error de temporizaci\'{o}n
\emph{non-data-aided} (NDA) que no requiere decisiones previas del
demodulador~\cite{gardner1986}.

\subsection{Algoritmo}

El error de temporizaci\'{o}n de Gardner para los canales I y Q se calcula
como:

\begin{equation}
    e[n] = \bigl(I[n{-}1] - I[n]\bigr) \cdot I_\text{mid}
         + \bigl(Q[n{-}1] - Q[n]\bigr) \cdot Q_\text{mid}
    \label{eq:gardner_err}
\end{equation}

donde $I[n]$,~$Q[n]$ son las muestras \emph{prompt} (instante \'{o}ptimo
actual), $I[n{-}1]$,~$Q[n{-}1]$ son los prompts del s\'{\i}mbolo
anterior, e $I_\text{mid}$,~$Q_\text{mid}$ son las muestras en el punto
medio entre ambos prompts.

La propiedad fundamental de este detector es que su valor esperado es
cero cuando el muestreo est\'{a} perfectamente alineado ($e[n] = 0$ en
media), y tiene un gradiente que permite al lazo de control converger al
instante \'{o}ptimo.

\subsection{NCO de temporizaci\'{o}n}

El control de fase temporal se implementa mediante un oscilador controlado
num\'{e}ricamente (NCO) de 16~bits.  El acumulador de fase se incrementa
por un paso nominal $\Delta\phi = 2^{16}/\text{SPS} = 16384$ en cada
muestra v\'{a}lida.  Cuando el acumulador desborda (bit~16), se genera un
pulso \texttt{sym\_strobe} que indica el instante \'{o}ptimo de muestreo.

El filtro de lazo PI ajusta el paso del NCO:

\begin{equation}
    \Delta\phi_\text{adj} = \Delta\phi_\text{nom}
        - K_p \cdot e[n]
        - K_i \cdot \sum_{k} e[k]
    \label{eq:ted_loop}
\end{equation}

con $K_p = 2^{-8}$ y $K_i = 2^{-16}$ (ganancias conservadoras
implementadas como desplazamientos aritm\'{e}ticos).

\subsection{Interpolaci\'{o}n lineal}

Para obtener la muestra \emph{prompt} en el instante exacto indicado
por el NCO (que generalmente no coincide con una muestra real), se emplea
un interpolador lineal:

\begin{equation}
    y(\mu) = x_\text{curr} + \mu \cdot (x_\text{prev} - x_\text{curr}),
    \qquad
    \mu = \frac{\text{NCO}_\text{overshoot}}{2^{14}}
    \label{eq:interp}
\end{equation}

donde $\mu \in [0, 1)$ es la fracci\'{o}n de muestra derivada de los bits
inferiores del acumulador NCO.

\paragraph{Justificaci\'{o}n del interpolador lineal.}
La Tang Nano~9K dispone de 20~multiplicadores DSP de
$9 \times 9$~bits.  La interpolaci\'{o}n lineal requiere solo
2~multiplicadores (uno por canal I/Q) para el producto
$\mu \times (x_\text{prev} - x_\text{curr})$, frente a los
6--8~multiplicadores que precisar\'{\i}a un interpolador c\'{u}bico o de
Farrow.  Dado que el error de interpolaci\'{o}n lineal afecta
principalmente a las componentes de alta frecuencia (atenuadas por el
filtro RRC con $\alpha = 0{,}25$), la degradaci\'{o}n de rendimiento es
insignificante para nuestra aplicaci\'{o}n.


% ============================================================================
\section{Recuperaci\'{o}n de Portadora --- Costas Loop 16-QAM}
\label{sec:costas}
% ============================================================================

La recuperaci\'{o}n de portadora corrige cualquier desfase de fase y/o
frecuencia residual entre el transmisor y el receptor.  El m\'{o}dulo
\texttt{costas\_loop} implementa un lazo de fase enganchada (\emph{PLL})
dirigido por decisiones (\emph{Decision-Directed}, DD), adaptado
espec\'{\i}ficamente para 16-QAM.

\subsection{Limitaci\'{o}n del Costas cl\'{a}sico con 16-QAM}
\label{sec:costas_clasico}

El detector de fase del lazo de Costas
cl\'{a}sico, utilizado habitualmente en BPSK y QPSK, calcula el error como:

\begin{equation}
    e_\text{QPSK} = Q \cdot \text{sgn}(I) - I \cdot \text{sgn}(Q)
    \label{eq:costas_qpsk}
\end{equation}

Este detector asume que la se\~{n}al solo tiene \emph{dos niveles por eje}
($\pm 1$).  En 16-QAM, con \emph{cuatro niveles por eje}
($\pm 1, \pm 3$ normalizados), el detector cl\'{a}sico falla porque:

\begin{itemize}
    \item Los s\'{\i}mbolos del anillo interior ($\pm 1/\!\sqrt{10}$)
          producen un gradiente de error \emph{ambiguo} respecto a los
          del anillo exterior ($\pm 3/\!\sqrt{10}$).
    \item La funci\'{o}n de error resultante (curva~S) presenta ceros
          espurios que impiden la convergencia del lazo.
\end{itemize}

\subsection{Soluci\'{o}n implementada: detector dirigido por decisiones}

En lugar del \texttt{sgn()}, se utiliza la salida del \emph{decisor duro}
(slicer) de 16-QAM para generar el error de fase mediante el producto
cruzado:

\begin{equation}
    e_\text{DD}[n] = \text{rot}_I \cdot \hat{Q} - \text{rot}_Q \cdot \hat{I}
    \label{eq:dd_error}
\end{equation}

donde $\text{rot}_{I,Q}$ son las muestras tras la rotaci\'{o}n del NCO,
y $\hat{I}$, $\hat{Q}$ son los niveles QAM m\'{a}s cercanos determinados
por las decisiones del slicer (umbral en $(\lvert\text{QAM}_{+1}\rvert
+ \lvert\text{QAM}_{+3}\rvert)/2 = 1296$ en Q1.11).

Este detector genera una curva~S correcta con ceros estables a $0\degree$,
$90\degree$, $180\degree$ y $270\degree$, y funciona para cualquier
constelaci\'{o}n QAM rectangular.

\subsection{Rotador de fase (NCO + LUT sin/cos)}

La rotaci\'{o}n de la constelaci\'{o}n se implementa mediante el producto
complejo:

\begin{align}
    \text{rot}_I &= \text{sym}_I \cdot \cos\theta - \text{sym}_Q \cdot \sin\theta \\
    \text{rot}_Q &= \text{sym}_I \cdot \sin\theta + \text{sym}_Q \cdot \cos\theta
    \label{eq:rotator}
\end{align}

donde $\theta$ es la fase acumulada del NCO de 16~bits.

\paragraph{LUT de cuarto de onda vs. CORDIC.}
Los valores de $\cos\theta$ y $\sin\theta$ se obtienen de una LUT de
cuarto de onda de 65~entradas en Q1.11, con plegado de cuadrante
(\emph{quadrant folding}) sobre los 2~bits superiores de la fase. Esta
implementaci\'{o}n se eligi\'{o} frente al algoritmo CORDIC por las
siguientes razones:

\begin{enumerate}
    \item \textbf{Uso eficiente de DSP:} La LUT produce valores
          $\sin/\cos$ en un ciclo de reloj, y las 4~multiplicaciones del
          rotador se realizan en multiplicadores DSP dedicados.
          La Tang Nano~9K dispone de 20~multiplicadores de $9\times 9$~bits
          (encadenables a $18\times 18$) que, en su mayor\'{\i}a, quedan
          sin usar despu\'{e}s del filtro FIR.  Aprovechar estos
          multiplicadores para el rotador ahorra l\'{o}gica combinacional
          (LUTs) frente a la cadena de sumas y desplazamientos del CORDIC.
    \item \textbf{Latencia reducida:} El rotador basado en LUT requiere
          solo 1--2~ciclos de \emph{pipeline}, mientras que un CORDIC de
          12~bits necesitar\'{\i}a 12~iteraciones (o un \emph{pipeline}
          de 12~etapas).
    \item \textbf{Memoria m\'{\i}nima:} $65 \times 12 = 780$~bits
          de LUT, distribuibles en l\'{o}gica sin consumir bloques BSRAM.
\end{enumerate}

\subsection{Filtro de lazo PI con \emph{gear shifting}}
\label{sec:loop_filter}

El filtro de lazo es de tipo~II (proporcional + integral), con ganancias
implementadas como desplazamientos aritm\'{e}ticos para evitar
multiplicadores adicionales.  Se emplea una t\'{e}cnica de
\emph{gear shifting}~\cite{meyr1997} que utiliza ganancias diferentes
durante la adquisici\'{o}n y el seguimiento:

\begin{table}[H]
\centering
\caption{Ganancias del filtro de lazo (dual \emph{gear shifting}).}
\label{tab:loop_gains}
\begin{tabular}{lccl}
\toprule
\textbf{Par\'{a}metro} & \textbf{Adquisici\'{o}n} & \textbf{Seguimiento}
    & \textbf{Raz\'{o}n} \\
\midrule
$K_p$ & $2^{-8} \approx 1/256$ & $2^{-3} \approx 1/8$
    & Fase: d\'{e}bil$\to$fuerte \\
$K_i$ & $2^{-7} \approx 1/128$ & $2^{-12} \approx 1/4096$
    & Frecuencia: fuerte$\to$congelado \\
\bottomrule
\end{tabular}
\end{table}

Durante la \textbf{adquisici\'{o}n} (primeros 200~s\'{\i}mbolos), el
integrador de frecuencia ($\omega$) domina con un $K_i$ elevado, permitiendo
que $\omega$ converja al offset de frecuencia real.  $K_p$ es d\'{e}bil
para no <<robar>> error de fase al integrador.

% After acquisition
Tras la \textbf{conmutaci\'{o}n} a modo de seguimiento, $K_i$ se reduce
drásticamente (el integrador queda pr\'{a}cticamente congelado al valor
adquirido), mientras que $K_p$ se incrementa para
corregir r\'{a}pidamente el residuo de fase resultante de la
cuantizaci\'{o}n entera de $\omega$.

Las ecuaciones de actualizaci\'{o}n del NCO son:

\begin{align}
    \omega[n+1] &= \omega[n] + K_i \cdot e[n]
        & & \text{(solo si } |e[n]| > \text{DEAD\_ZONE)} \\
    \theta[n+1] &= \theta[n] + \omega[n] + K_p \cdot e[n]
        & & \text{(siempre activo tras \emph{holdoff})}
    \label{eq:nco_update}
\end{align}

\paragraph{Zona muerta.}
El integrador emplea una zona muerta
($|\,e\,| > 50$~LSBs) para evitar que el ruido de cuantizaci\'{o}n acumule
una desviaci\'{o}n sistem\'{a}tica en $\omega$.  La ruta proporcional
opera \emph{sin} zona muerta (los errores peque\~{n}os se truncan
naturalmente a $K_p \cdot e = 0$).

\paragraph{Correcci\'{o}n de truncamiento asim\'{e}trico.}
El desplazamiento aritm\'{e}tico a la derecha (\texttt{>>>}) en Verilog
redondea hacia $-\infty$, lo que introduce un sesgo negativo:
$-69 \ggg 6 = -2$ pero $+69 \ggg 6 = +1$.  Para evitar la
deriva permanente del NCO, se implementa un truncamiento sim\'{e}trico
hacia cero:

\begin{equation}
    \text{trunc}(x, s) = \begin{cases}
        x \ggg s & x \geq 0 \\
        -((-x) \ggg s) & x < 0
    \end{cases}
    \label{eq:trunc_zero}
\end{equation}

\subsection{Detector de enganche}
\label{sec:lock_det}

El detector de enganche calcula una media m\'{o}vil exponencial (EMA) del
valor absoluto del error de fase:

\begin{equation}
    \bar{e}[n] = \bar{e}[n{-}1]
        + \frac{|e[n]| - \bar{e}[n{-}1]}{2^6}
    \label{eq:ema}
\end{equation}

Se declara enganche (\texttt{demod\_lock}~$= 1$) cuando
$\bar{e} < 150$ y han transcurrido al menos 100~s\'{\i}mbolos
(\emph{holdoff} para evitar falsos enganches durante el transitorio
inicial).


% ============================================================================
\section{Presupuesto de Recursos DSP}
\label{sec:dsp_budget}
% ============================================================================

La Tabla~\ref{tab:dsp} resume la asignaci\'{o}n de multiplicadores DSP
del subsistema~RX.  Todos operan a tasa de s\'{\i}mbolo
($\SI{6,75}{\mega\hertz}$) excepto los del filtro acoplado y la
compensaci\'{o}n de ganancia, que operan a tasa de muestra
($\SI{27}{\mega\hertz}$).

\begin{table}[H]
\centering
\caption{Multiplicadores DSP utilizados por el subsistema~RX.}
\label{tab:dsp}
\begin{tabular}{lccl}
\toprule
\textbf{M\'{o}dulo} & \textbf{Mults.} & \textbf{Tasa} & \textbf{Operaci\'{o}n} \\
\midrule
Filtro RRC (I)            &     2    & muestra & MAC FIR (5 taps)\\
Filtro RRC (Q)            &     2    & muestra & MAC FIR (5 taps) \\
Interpolador Gardner      &     2    & s\'{\i}mbolo & $\mu \times \Delta x$ \\
Error Gardner             &     2    & s\'{\i}mbolo & $(I_\text{prev}-I)\cdot I_\text{mid}$ \\
Rotador Costas            &     4    & s\'{\i}mbolo & $I\cos$, $Q\sin$, etc. \\
Error Costas DD           &     2    & s\'{\i}mbolo & $I \hat{Q} - Q \hat{I}$ \\
\midrule
\textbf{Total RX}         &  \textbf{14} & & (compartido con TX: 10 DSP slices) \\
\bottomrule
\end{tabular}
\end{table}

Dado que los multiplicadores de tasa de s\'{\i}mbolo pueden compartirse
temporalmente con los de tasa de muestra (mediante \emph{time-division
multiplexing}), el consumo efectivo se mantiene dentro de los
20~bloques DSP disponibles en la GW1NR-9.


% ============================================================================
\section{Resultados de Verificaci\'{o}n}
\label{sec:results}
% ============================================================================

\subsection{Prueba funcional b\'{a}sica}

La primera validaci\'{o}n ejecuta la cadena completa
TX$\to$Canal$\to$RX sin offset de frecuencia (escenario ideal).
Con ruido AWGN ($M = 24$, SNR~$\approx \SI{20}{\decibel}$):

\begin{itemize}
    \item Enganche alcanzado en el s\'{\i}mbolo~205.
    \item Precisi\'{o}n post-enganche: 100\,\% (500/500 s\'{\i}mbolos
          dentro de tolerancia $\pm 300$~LSBs).
    \item Error m\'{a}ximo I/Q: 176/153~LSBs.
    \item NCO estable en \texttt{0x0000} (sin offset, no se requiere
          correcci\'{o}n).
\end{itemize}

\subsection{Prueba de estr\'{e}s con offset de frecuencia}
\label{sec:stress}

Para validar que el lazo de Costas puede adquirir y seguir un offset de
frecuencia real, se inyecta una rotaci\'{o}n compleja a la salida del
transmisor:

\begin{equation}
    r_\text{rot}[n] = r_\text{TX}[n] \cdot e^{j\,2\pi f_\Delta n/f_s}
    \label{eq:cfo}
\end{equation}

con $f_\Delta = \SI{5}{\kilo\hertz}$, equivalente a una rotaci\'{o}n de
$\approx 0{,}27\degree$ por s\'{\i}mbolo.

Resultados de la prueba de estr\'{e}s (AWGN $M=24$ + CFO
$\SI{5}{\kilo\hertz}$):

\begin{itemize}
    \item Enganche alcanzado en el s\'{\i}mbolo~326.
    \item Precisi\'{o}n post-enganche: 81\,\% (407/500), por encima del
          umbral m\'{\i}nimo del 80\,\%.
    \item $\omega$ (estimaci\'{o}n de frecuencia del integrador):
          converge a $-48$ unidades NCO/s\'{\i}mbolo, consistente con el
          valor te\'{o}rico
          $\omega_\text{ideal} = f_\Delta \cdot \text{SPS} \cdot 2^{16}
          / f_s \approx -48{,}6$.
    \item NCO activo: la fase del NCO rota continuamente para compensar
          el offset inyectado.
    \item \textbf{Veredicto: \texttt{Stress Test Passed: NCO active and
          correcting.}}
\end{itemize}

La precisi\'{o}n del 81\,\% (frente al 100\,\% sin CFO) se debe a la
cuantizaci\'{o}n entera de $\omega$: el valor $-48$ difiere del ideal
$-48{,}6$ en $\approx 1{,}2\,\%$, generando un peque\~{n}o residuo de
fase oscilante que la ruta proporcional corrige de forma c\'{\i}clica.
Este efecto es inherente a un NCO de 16~bits y podr\'{\i}a eliminarse
aumentando la anchura del acumulador a 24--32~bits, como se realiza en
implementaciones comerciales.


% ============================================================================
\section{Resultados de Simulaci\'{o}n del Costas Loop}
\label{sec:costas_results}
% ============================================================================

Las siguientes figuras muestran el comportamiento del lazo Costas bajo
diferentes condiciones de fase inicial, generadas por el script de
depuraci\'{o}n \texttt{debug\_rx\_chain.py}.

\begin{figure}[H]
    \centering
    \includegraphics[width=0.65\textwidth]{../figures/debug_01b_after.png}
    \caption{Constelaci\'{o}n tras el lazo Costas sin impairments
             (fase inicial 0\textdegree). Los 16 s\'{\i}mbolos est\'{a}n claramente
             separados, demostrando convergencia correcta.}
    \label{fig:costas_0deg}
\end{figure}

\begin{figure}[H]
    \centering
    \includegraphics[width=0.65\textwidth]{../figures/debug_02b_after.png}
    \caption{Constelaci\'{o}n tras el lazo Costas con 45\textdegree~de offset
             de fase inicial. El lazo corrige la rotaci\'{o}n llevando
             los s\'{\i}mbolos a sus posiciones ideales.}
    \label{fig:costas_45deg}
\end{figure}

\begin{figure}[H]
    \centering
    \includegraphics[width=0.65\textwidth]{../figures/debug_03b_after.png}
    \caption{Constelaci\'{o}n tras el lazo Costas con 90\textdegree~de offset
             de fase inicial, dentro del rango de ambig\"{u}edad de 16-QAM.
             El lazo converge a una de las cuatro orientaciones v\'{a}lidas.}
    \label{fig:costas_90deg}
\end{figure}


% ============================================================================
\section{Conclusiones de la Fase~3}
\label{sec:conclusions}
% ============================================================================

Se ha implementado y verificado el subsistema de recepci\'{o}n completo
del m\'{o}dem 16-QAM G-DSP Engine.  Los principales logros son:

\begin{enumerate}
    \item \textbf{Filtro acoplado:} Reutilizaci\'{o}n del m\'{o}dulo
          RRC existente con compensaci\'{o}n de ganancia calibrada.
    \item \textbf{Gardner TED:} Recuperaci\'{o}n de temporizaci\'{o}n
          NDA con interpolaci\'{o}n lineal, optimizada para los recursos
          limitados de la Tang Nano~9K.
    \item \textbf{Costas DD-PLL:} Recuperaci\'{o}n de portadora
          adaptada a 16-QAM, con LUT de cuarto de onda, lazo PI con
          \emph{gear shifting}, truncamiento sim\'{e}trico y detector
          de enganche.
    \item \textbf{Verificaci\'{o}n:} Prueba funcional (100\,\% de
          precisi\'{o}n sin CFO) y prueba de estr\'{e}s (81\,\% con
          $\SI{5}{\kilo\hertz}$ de offset) superadas satisfactoriamente,
          demostrando que el NCO responde activamente para compensar
          la rotaci\'{o}n inyectada.
\end{enumerate}


% ============================================================================
% Referencias
% ============================================================================
\begin{thebibliography}{9}

\bibitem{haykin2014}
S.~Haykin, \emph{Communication Systems}, 5th ed. Hoboken, NJ: Wiley, 2014.

\bibitem{gardner1986}
F.~M.~Gardner, ``A BPSK/QPSK timing-error detector for sampled
receivers,'' \emph{IEEE Trans.\ Commun.}, vol.~COM-34, no.~5,
pp.~423--429, May~1986.

\bibitem{meyr1997}
H.~Meyr, M.~Moeneclaey, and S.~A.~Fechtel, \emph{Digital Communication
Receivers: Synchronization, Channel Estimation, and Signal Processing}.
New York: Wiley, 1997.

\end{thebibliography}

\end{document}
