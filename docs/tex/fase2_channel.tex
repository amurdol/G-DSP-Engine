% !TEX root = fase2_channel.tex
% ============================================================================
% G-DSP Engine --- Documentaci\'{o}n T\'{e}cnica: Fase 2
% Simulaci\'{o}n de Canal AWGN
% ============================================================================
% Compilar:  pdflatex fase2_channel.tex  (x2 para referencias)
% ============================================================================

\documentclass[a4paper,11pt,spanish]{article}

% ---------------------------------------------------------------------------
% Paquetes
% ---------------------------------------------------------------------------
\usepackage[utf8]{inputenc}
\usepackage[T1]{fontenc}
\usepackage[spanish,es-tabla]{babel}
\usepackage{amsmath,amssymb,amsfonts}
\usepackage{booktabs}
\usepackage{array}
\usepackage{graphicx}
\usepackage{xcolor}
\usepackage{listings}
\usepackage{siunitx}
\usepackage{hyperref}
\usepackage{geometry}
\usepackage{caption}
\usepackage{subcaption}
\usepackage{enumitem}
\usepackage{float}
\usepackage{tikz}
\usetikzlibrary{arrows.meta, positioning, calc, shapes.geometric, patterns, babel}

\geometry{margin=2.5cm}

\sisetup{
    output-decimal-marker = {,},
    per-mode = symbol
}

% ---------------------------------------------------------------------------
% Configuraci\'{o}n de listings para SystemVerilog
% ---------------------------------------------------------------------------
\lstdefinelanguage{SystemVerilog}{
    morekeywords={module, endmodule, input, output, logic, signed, parameter,
                  int, always_ff, always_comb, posedge, negedge, if, else,
                  begin, end, for, case, endcase, function, automatic,
                  endfunction, localparam, assign, typedef, package,
                  endpackage, import, default, generate, endgenerate,
                  genvar},
    sensitive=true,
    morecomment=[l]{//},
    morecomment=[s]{/*}{*/},
    morestring=[b]",
}

\lstset{
    language=SystemVerilog,
    basicstyle=\ttfamily\footnotesize,
    keywordstyle=\bfseries\color{blue!70!black},
    commentstyle=\itshape\color{green!50!black},
    stringstyle=\color{red!60!black},
    numbers=left,
    numberstyle=\tiny\color{gray},
    numbersep=5pt,
    frame=single,
    breaklines=true,
    captionpos=b,
    tabsize=4,
    showstringspaces=false,
}

% ---------------------------------------------------------------------------
% Datos del documento
% ---------------------------------------------------------------------------
\title{%
    \textbf{G-DSP Engine} \\[0.5em]
    \Large Simulaci\'{o}n de Canal AWGN \\[0.3em]
    \large Fase~2 --- Documentaci\'{o}n T\'{e}cnica del TFG
}
\author{G-DSP Team}
\date{Febrero 2026}

% ============================================================================
\begin{document}
% ============================================================================

\maketitle
\tableofcontents
\newpage

% ============================================================================
\section{Introducci\'{o}n}
\label{sec:ch_intro}
% ============================================================================

El modelado del canal de comunicaciones es un componente esencial en
cualquier sistema de verificaci\'{o}n de m\'{o}dem. En un escenario de
laboratorio cl\'{a}sico, el ruido del canal se introduce externamente
mediante un generador de ruido calibrado; sin embargo, para una
demostraci\'{o}n aut\'{o}noma sobre FPGA ---~como la que persigue el
proyecto G-DSP Engine~--- resulta imprescindible sintetizar dicho ruido
\emph{dentro} de la propia l\'{o}gica programable.

El objetivo de la Fase~2 es implementar un \textbf{canal AWGN}
(\textit{Additive White Gaussian Noise}) que permita:

\begin{enumerate}[label=\textbf{\arabic*.}]
    \item Inyectar ruido de distribuci\'{o}n aproximadamente gaussiana
          a las se\~{n}ales $I$ y $Q$ procedentes del transmisor.
    \item Controlar la potencia del ruido en tiempo real mediante un
          registro de magnitud (\texttt{noise\_magnitude}), facilitando
          barridos de $E_b/N_0$ para curvas de BER.
    \item Garantizar la independencia estad\'{\i}stica entre los canales
          $I$ y $Q$.
\end{enumerate}

La cadena completa queda:

\begin{equation}
    r_{I,Q}[n] = s_{I,Q}[n] + w_{I,Q}[n]
    \label{eq:channel_model}
\end{equation}

\noindent
donde $s_{I,Q}$ es la se\~{n}al transmitida (Q1.11) y $w_{I,Q}$ es el
ruido gaussiano generado en hardware.


% ============================================================================
\section{Justificaci\'{o}n de Arquitectura: CLT frente a Box-Muller}
\label{sec:justificacion}
% ============================================================================

Se evaluaron tres estrategias para la generaci\'{o}n de muestras con
distribuci\'{o}n gaussiana en hardware:

% ----------------------------------------------------------------------------
\subsection{Transformada de Box-Muller}
\label{subsec:box_muller}
% ----------------------------------------------------------------------------

La transformada de Box-Muller genera pares de variables gaussianas
independientes a partir de dos variables uniformes $U_1, U_2 \in (0,1)$:

\begin{equation}
    Z_0 = \sqrt{-2\ln U_1}\,\cos(2\pi U_2), \qquad
    Z_1 = \sqrt{-2\ln U_1}\,\sin(2\pi U_2)
    \label{eq:box_muller}
\end{equation}

\paragraph{Coste de implementaci\'{o}n.}
El c\'{a}lculo de $\ln(\cdot)$ y $\sqrt{\cdot}$ en hardware requiere
aproximaciones por CORDIC o tablas de consulta (LUT) almacenadas en BSRAM.
Las funciones trigonom\'{e}tricas a\~{n}aden un segundo bloque CORDIC.
La estimaci\'{o}n de recursos para la FPGA GW1NR-9 es:

\begin{table}[H]
    \centering
    \caption{Coste estimado de Box-Muller en GW1NR-9.}
    \label{tab:box_muller_cost}
    \begin{tabular}{@{} l c c @{}}
        \toprule
        \textbf{Recurso} & \textbf{Necesario} & \textbf{Disponible} \\
        \midrule
        LUT4            & $>\num{1200}$   & \num{8640} \\
        BSRAM (18~Kbit) & 2--4 bloques    & 26 bloques \\
        DSP (MULT18)    & 2--4            & 10 \\
        Latencia        & 15--20 ciclos   & --- \\
        \bottomrule
    \end{tabular}
\end{table}

\noindent
El consumo de LUTs y BSRAM es significativo y compite directamente con
los recursos necesarios para el filtro RRC, la cadena de sincronizaci\'{o}n
y el pipeline HDMI. \textbf{Se descarta} esta opci\'{o}n.

% ----------------------------------------------------------------------------
\subsection{M\'{e}todo del Zigurat}
\label{subsec:ziggurat}
% ----------------------------------------------------------------------------

El m\'{e}todo del zigurat es una t\'{e}cnica de rechazo
(\textit{rejection sampling}) que encapsula la PDF gaussiana en
rect\'{a}ngulos apilados. Su principal inconveniente para hardware es
la \textbf{latencia variable}: cada muestra requiere un n\'{u}mero
indeterminado de iteraciones del bucle de rechazo, lo que rompe el
determinismo del pipeline. Se necesitar\'{\i}a un FIFO de
desacoplamiento y l\'{o}gica de control adicional.
\textbf{Se descarta} por complejidad.

% ----------------------------------------------------------------------------
\subsection{Teorema del L\'{\i}mite Central (CLT) --- Opci\'{o}n seleccionada}
\label{subsec:clt}
% ----------------------------------------------------------------------------

El Teorema del L\'{\i}mite Central establece que la suma de $N$ variables
aleatorias independientes e id\'{e}nticamente distribuidas (i.i.d.)
converge en distribuci\'{o}n a una variable gaussiana conforme $N$ crece:

\begin{equation}
    Z = \frac{\displaystyle\sum_{i=1}^{N} X_i - N\mu}{\sigma\sqrt{N}}
    \;\xrightarrow{\;d\;}\;
    \mathcal{N}(0,\,1)
    \quad\text{cuando } N \to \infty
    \label{eq:clt}
\end{equation}

\noindent
donde $X_i \sim \text{Uniforme}[0, 2^{12})$, $\mu = 2^{11} - 0{,}5$
y $\sigma^2 = (2^{12})^2 / 12$.

Para $N = 16$, la calidad de la aproximaci\'{o}n se eval\'{u}a mediante
la curtosis en exceso:

\begin{equation}
    \kappa_{\text{exc}} = \frac{\kappa_4}{\sigma^4} - 3
    = \frac{-6/5}{N}
    = \frac{-6/5}{16}
    = -0{,}075
    \label{eq:kurtosis}
\end{equation}

\noindent
Un valor de $|\kappa_{\text{exc}}| = 0{,}075$ frente a los 0 de una
gaussiana pura indica que las colas de la distribuci\'{o}n son
ligeramente m\'{a}s ligeras. En la pr\'{a}ctica, esta desviaci\'{o}n
es indistinguible para pruebas de BER con $E_b/N_0 \geq \SI{4}{\decibel}$.

\paragraph{Coste de implementaci\'{o}n.}

\begin{table}[H]
    \centering
    \caption{Coste estimado de CLT ($N=16$) en GW1NR-9.}
    \label{tab:clt_cost}
    \begin{tabular}{@{} l c l @{}}
        \toprule
        \textbf{Recurso} & \textbf{Estimaci\'{o}n} & \textbf{Justificaci\'{o}n} \\
        \midrule
        LUT4  & $\sim$350--500 & 16 LFSRs + \'{a}rbol de sumadores \\
        BSRAM & 0              & Sin tablas de consulta \\
        DSP   & 1 (MULT18)     & Multiplicaci\'{o}n por \texttt{noise\_magnitude} \\
        FF    & $\sim$400      & 16 registros LFSR + pipeline \\
        Latencia & 3 ciclos    & Determinista, sin rechazo \\
        \bottomrule
    \end{tabular}
\end{table}

\noindent
El coste es una fracci\'{o}n menor de los recursos de la FPGA: sin BSRAM,
sin CORDIC, sin latencia variable. \textbf{Se selecciona} esta
arquitectura.


% ============================================================================
\section{Fundamento Matem\'{a}tico}
\label{sec:math}
% ============================================================================

% ----------------------------------------------------------------------------
\subsection{Distribuci\'{o}n resultante de la suma de $N=16$ uniformes}
\label{subsec:irwin_hall}
% ----------------------------------------------------------------------------

Sea $X_i$ la salida de 12~bits sin signo del $i$-\'{e}simo LFSR, con
distribuci\'{o}n uniforme discreta en $\{0, 1, \ldots, 2^{12}-1\}$. La
suma $S = \sum_{i=1}^{16} X_i$ sigue (para $N$ grande) una distribuci\'{o}n
de Irwin-Hall, cuya PDF converge a la campana gaussiana.

Los par\'{a}metros de $S$ son:

\begin{align}
    \text{Media:}    \quad & \mu_S  = N \cdot \frac{2^{12}-1}{2}
                           = 16 \times 2047{,}5 = 32760
    \label{eq:mean_S} \\
    \text{Varianza:} \quad & \sigma_S^2 = N \cdot \frac{(2^{12})^2}{12}
                           = 16 \times \frac{16\,777\,216}{12}
                           \approx \num{22369621}
    \label{eq:var_S}
\end{align}

Se centra la distribuci\'{o}n sustrayendo $2^{15} = 32768$ (potencia de~2
m\'{a}s cercana a $\mu_S$, implementable como simple inversi\'{o}n de
bits), obteni\'{e}ndose un valor con signo de 17~bits.

% ----------------------------------------------------------------------------
\subsection{Relaci\'{o}n entre \texttt{noise\_magnitude} y SNR}
\label{subsec:mag_to_snr}
% ----------------------------------------------------------------------------

Sea $M \in [0, 255]$ el valor del registro \texttt{noise\_magnitude} y
$w[n]$ la muestra de ruido a la salida del generador (formato Q1.11).
El procesamiento interno es:

\begin{equation}
    w[n] = \frac{(S[n] - 2^{15}) \times M}{2^{12}}
    \label{eq:noise_sample}
\end{equation}

\noindent
donde la divisi\'{o}n por $2^{12} = 2^4 \times 2^8$ corresponde a:
\begin{itemize}
    \item $\div\, 2^4$: normalizaci\'{o}n de la suma (dividir entre $N=16$
          para obtener varianza unitaria referida a la escala uniforme).
    \item $\div\, 2^8$: escalado por el factor $M/256$.
\end{itemize}

La varianza de $w[n]$ resulta:

\begin{align}
    \sigma_w^2 &= \left(\frac{M}{2^{12}}\right)^2 \cdot \sigma_S^2
    = \left(\frac{M}{4096}\right)^2 \cdot 16 \cdot \frac{(2^{12})^2}{12}
    \nonumber \\
    &= \frac{M^2}{4096^2} \cdot \frac{16 \cdot 4096^2}{12}
    = \frac{M^2 \cdot 16}{12}
    = \frac{4\,M^2}{3}
    \label{eq:sigma_w}
\end{align}

Expresando en unidades Q1.11 (donde el fondo de escala es $2^{11} = 2048$),
la varianza normalizada es:

\begin{equation}
    \sigma_{w,\text{norm}}^2
    = \frac{\sigma_w^2}{(2^{11})^2}
    = \frac{4\,M^2}{3 \times 2^{22}}
    = \frac{M^2}{3 \times 2^{20}}
    \approx \frac{M^2}{\num{3145728}}
    \label{eq:sigma_norm}
\end{equation}

Para la constelaci\'{o}n 16-QAM normalizada a potencia unitaria
($P_s = 1$), la relaci\'{o}n se\~{n}al a ruido considerando ambos
canales ($I$ y $Q$, con ruido independiente) es:

\begin{equation}
    \text{SNR} = \frac{P_s}{2\,\sigma_{w,\text{norm}}^2}
    = \frac{3 \times 2^{20}}{2\,M^2}
    \label{eq:snr_linear}
\end{equation}

En decibelios:

\begin{equation}
    \boxed{
    \text{SNR}_{\text{dB}} = 10\log_{10}\!\left(\frac{3 \times 2^{20}}{2\,M^2}\right)
    \approx 59{,}0 - 20\log_{10}(M)
    }
    \label{eq:snr_db}
\end{equation}

La Tabla~\ref{tab:snr_mapping} presenta valores representativos.

\begin{table}[H]
    \centering
    \caption{Mapeo de \texttt{noise\_magnitude} ($M$) a SNR.}
    \label{tab:snr_mapping}
    \begin{tabular}{@{} c r r l @{}}
        \toprule
        $M$ & $\sigma_{w,\text{norm}}$ & $\text{SNR}_{\text{dB}}$
            & \textbf{Observaci\'{o}n} \\
        \midrule
        0   & 0        & $\infty$ & Sin ruido (bypass) \\
        4   & 0,0011   & 46,9     & Pr\'{a}cticamente limpio \\
        16  & 0,0045   & 34,9     & Ruido imperceptible \\
        32  & 0,0090   & 28,8     & Leve dispersi\'{o}n \\
        64  & 0,0181   & 22,8     & Nube visible \\
        128 & 0,0362   & 16,8     & Constelaci\'{o}n degradada \\
        200 & 0,0565   & 12,9     & L\'{\i}mite del m\'{o}dem \\
        255 & 0,0720   & 10,8     & Ruido m\'{a}ximo \\
        \bottomrule
    \end{tabular}
\end{table}


% ============================================================================
\section{Detalles de Implementaci\'{o}n RTL}
\label{sec:impl}
% ============================================================================

% ----------------------------------------------------------------------------
\subsection{Banco de 16 LFSRs con polinomios primitivos verificados}
\label{subsec:lfsr_bank}
% ----------------------------------------------------------------------------

La calidad de la aproximaci\'{o}n CLT depende cr\'{\i}ticamente de que
las 16 fuentes de ruido uniforme sean \textbf{estad\'{\i}sticamente
independientes}. Esto se garantiza mediante tres mecanismos:

\paragraph{1. Polinomios primitivos distintos.}
Cada LFSR utiliza un polinomio trinomial $x^n + x^k + 1$ cuya
primitividad ha sido verificada a partir de las tablas de referencia
de Xilinx (XAPP052, P.~Alfke, 1996). Se emplean 11~anchuras distintas
(de 15 a 31~bits) y 5~polinomios rec\'{\i}procos adicionales
($x^n + x^{n-k} + 1$). La Tabla~\ref{tab:lfsr_polys} recoge el
conjunto completo.

\begin{table}[H]
    \centering
    \caption{Polinomios primitivos de los 16 LFSRs de ruido.}
    \label{tab:lfsr_polys}
    \small
    \begin{tabular}{@{} c c c r l @{}}
        \toprule
        \textbf{\#} & \textbf{Anchura $n$} & \textbf{Polinomio}
                    & \textbf{Per\'{\i}odo} & \textbf{Nota} \\
        \midrule
         0 & 15 & $x^{15}+x^{14}+1$ &      \num{32767} & \\
         1 & 17 & $x^{17}+x^{14}+1$ &     \num{131071} & \\
         2 & 18 & $x^{18}+x^{11}+1$ &     \num{262143} & \\
         3 & 20 & $x^{20}+x^{17}+1$ &   \num{1048575} & \\
         4 & 21 & $x^{21}+x^{19}+1$ &   \num{2097151} & \\
         5 & 22 & $x^{22}+x^{21}+1$ &   \num{4194303} & \\
         6 & 23 & $x^{23}+x^{18}+1$ &   \num{8388607} & ITU-T O.151 \\
         7 & 25 & $x^{25}+x^{22}+1$ &  \num{33554431} & \\
         8 & 28 & $x^{28}+x^{25}+1$ & \num{268435455} & \\
         9 & 29 & $x^{29}+x^{27}+1$ & \num{536870911} & \\
        10 & 31 & $x^{31}+x^{28}+1$ & \num{2147483647} & \\
        11 & 15 & $x^{15}+x^{1}+1$  &      \num{32767} & Rec\'{\i}proco de \#0 \\
        12 & 17 & $x^{17}+x^{3}+1$  &     \num{131071} & Rec\'{\i}proco de \#1 \\
        13 & 20 & $x^{20}+x^{3}+1$  &   \num{1048575} & Rec\'{\i}proco de \#3 \\
        14 & 23 & $x^{23}+x^{5}+1$  &   \num{8388607} & Rec\'{\i}proco de \#6 \\
        15 & 25 & $x^{25}+x^{3}+1$  &  \num{33554431} & Rec\'{\i}proco de \#7 \\
        \bottomrule
    \end{tabular}
\end{table}

\noindent
\textbf{Nota sobre anchuras sin trinomio primitivo.}
Las longitudes $n \in \{16, 19, 24, 26, 27, 30\}$ carecen de
trinomios primitivos de la forma $x^n + x^k + 1$ en
$\text{GF}(2)$~\cite{trinomial_table}. En la versi\'{o}n inicial del
dise\~{n}o se inclu\'{\i}an estas longitudes; se detectaron durante la
revisi\'{o}n t\'{e}cnica y se sustituyeron por pares rec\'{\i}procos de
longitudes v\'{a}lidas. Dos LFSRs de la \emph{misma} anchura pero con
polinomios rec\'{\i}procos generan secuencias distintas (una es la
inversi\'{o}n temporal de la otra), por lo que mantienen la
independencia.

\paragraph{2. Semillas (seeds) \'{u}nicas por LFSR y por instancia.}
Cada LFSR recibe una semilla calculada combinando:
\begin{itemize}
    \item El \'{\i}ndice $g \in [0,15]$ del LFSR dentro del generador.
    \item Un par\'{a}metro \texttt{INSTANCE\_ID} que distingue entre la
          instancia del canal~$I$ (\texttt{ID=0}) y la del canal~$Q$
          (\texttt{ID=1}).
\end{itemize}

La expresi\'{o}n de c\'{a}lculo de la semilla es:

\begin{equation}
    \text{seed}(g, \text{ID}) = \bigl[
        (\texttt{0xDEAD\_BEE0} + g \cdot \texttt{0x1357\_9BDF})
        \oplus (\texttt{0xA5A5\_A5A5} \gg g)
        \oplus (\text{ID} \cdot \texttt{0x5A5A\_DEAD})
    \bigr] \,|\, 1
    \label{eq:seed}
\end{equation}

\noindent
El OR final con~1 impide el estado de bloqueo ($s = 0$) inherente a
todo LFSR est\'{a}ndar. El t\'{e}rmino dependiente de \texttt{INSTANCE\_ID}
garantiza que los 16~LFSRs de la instancia $I$ produzcan secuencias
completamente distintas a los 16~LFSRs de la instancia $Q$, previniendo
la correlaci\'{o}n cruzada entre canales.

\paragraph{3. Bug corregido: correlaci\'{o}n I/Q.}
En la versi\'{o}n inicial del m\'{o}dulo \texttt{channel\_top}, ambas
instancias del generador de ruido (\texttt{u\_noise\_I} y
\texttt{u\_noise\_Q}) se instanciaban sin par\'{a}metro diferenciador.
Al compartir \texttt{INSTANCE\_ID = 0}, las 32~semillas resultantes
eran id\'{e}nticas, produciendo ruidos $w_I[n] = w_Q[n]$ perfectamente
correlacionados. Esto \textbf{invalida} la hip\'{o}tesis de canal AWGN,
que exige $E[w_I \cdot w_Q] = 0$.

La correcci\'{o}n consiste en parametrizar la instanciaci\'{o}n:

\begin{lstlisting}[caption={Instanciaci\'{o}n corregida con \texttt{INSTANCE\_ID} distinto.},
                   label={lst:instance_id}]
awgn_generator #(.INSTANCE_ID(0)) u_noise_I ( ... );
awgn_generator #(.INSTANCE_ID(1)) u_noise_Q ( ... );
\end{lstlisting}

% ----------------------------------------------------------------------------
\subsection{\'{A}rbol de sumadores}
\label{subsec:adder_tree}
% ----------------------------------------------------------------------------

Las 16 salidas uniformes de 12~bits se suman mediante un \'{a}rbol de
sumadores combinacional de 4 niveles:

\begin{equation}
    \underbrace{16 \to 8}_{\text{Nivel 0}}
    \;\to\;
    \underbrace{8 \to 4}_{\text{Nivel 1}}
    \;\to\;
    \underbrace{4 \to 2}_{\text{Nivel 2}}
    \;\to\;
    \underbrace{2 \to 1}_{\text{Nivel 3}}
    \label{eq:adder_tree}
\end{equation}

Cada sumador opera con operandos de $\text{DATA\_WIDTH} + 4 = 16$~bits
(extensi\'{o}n con ceros para capturar el acarreo). El resultado final
es un valor sin signo de 16~bits que se centra y registra en un flip-flop
(primera etapa del pipeline).

\begin{figure}[H]
    \centering
    \begin{tikzpicture}[
        lfsr/.style={rectangle, draw, fill=blue!10, minimum width=10mm,
                     minimum height=6mm, font=\scriptsize},
        add/.style={circle, draw, fill=orange!15, minimum size=5mm,
                    inner sep=0pt, font=\scriptsize},
        reg/.style={rectangle, draw, fill=green!10, minimum width=12mm,
                    minimum height=6mm, font=\scriptsize},
        >=Stealth,
        node distance=4mm and 8mm
    ]
        % LFSRs (simplified: show 4 of 16)
        \foreach \i in {0,...,3} {
            \node[lfsr] (L\i) at (0, -\i*0.9) {LFSR \i};
        }
        \node at (0, -4*0.9) {\vdots};
        \foreach \i/\j in {12/4, 13/5, 14/6, 15/7} {
            \node[lfsr] (L\i) at (0, -\j*0.9 - 1.0) {LFSR \i};
        }

        % Level 0 adders (show 2 of 8)
        \node[add] (A00) at (2.5, -0.45) {$+$};
        \node[add] (A01) at (2.5, -1.35) {$+$};
        \node at (2.5, -2.5) {\vdots};
        \node[add] (A06) at (2.5, -4.4) {$+$};
        \node[add] (A07) at (2.5, -5.3) {$+$};

        % Level 1 (show 2 of 4)
        \node[add] (A10) at (4.2, -0.9) {$+$};
        \node at (4.2, -2.5) {\vdots};
        \node[add] (A13) at (4.2, -4.85) {$+$};

        % Level 2 (2)
        \node[add] (A20) at (5.8, -1.5) {$+$};
        \node[add] (A21) at (5.8, -4.2) {$+$};

        % Level 3 (1)
        \node[add] (A30) at (7.2, -2.85) {$+$};

        % Pipeline register
        \node[reg] (R1) at (9.0, -2.85) {$z^{-1}$};

        % Multiply
        \node[add, fill=red!15] (MUL) at (10.8, -2.85) {$\times$};
        \node[below=1mm of MUL, font=\scriptsize] {$M$};

        % Output register
        \node[reg] (R2) at (12.3, -2.85) {$z^{-1}$};
        \node[right=4mm of R2, font=\footnotesize] {$w[n]$};

        % Connections (simplified)
        \draw[->] (L0.east) -- (A00);
        \draw[->] (L1.east) -- (A00);
        \draw[->] (L2.east) -- (A01);
        \draw[->] (L3.east) -- (A01);
        \draw[->] (L12.east) -- (A06);
        \draw[->] (L13.east) -- (A06);
        \draw[->] (L14.east) -- (A07);
        \draw[->] (L15.east) -- (A07);
        \draw[->] (A00) -- (A10);
        \draw[->] (A01) -- (A10);
        \draw[->] (A06) -- (A13);
        \draw[->] (A07) -- (A13);
        \draw[->] (A10) -- (A20);
        \draw[->] (A13) -- (A21);
        \draw[->] (A20) -- (A30);
        \draw[->] (A21) -- (A30);
        \draw[->] (A30) -- node[above, font=\scriptsize]{$-2^{15}$} (R1);
        \draw[->] (R1) -- (MUL);
        \draw[->] (MUL) -- node[above, font=\scriptsize]{$\gg 12$} (R2);
        \draw[->] (R2) -- ++(7mm,0);
    \end{tikzpicture}
    \caption{Pipeline del generador AWGN: 16~LFSRs, \'{a}rbol de sumadores
             de 4 niveles, centrado, escalado por~$M$ y truncaci\'{o}n.}
    \label{fig:awgn_pipeline}
\end{figure}

% ----------------------------------------------------------------------------
\subsection{Saturaci\'{o}n aritm\'{e}tica en la suma se\~{n}al + ruido}
\label{subsec:saturation}
% ----------------------------------------------------------------------------

La operaci\'{o}n central del canal es la adici\'{o}n:

\begin{equation}
    r[n] = s[n] + w[n]
    \label{eq:add}
\end{equation}

\noindent
donde tanto $s$ como $w$ son valores Q1.11 de 12~bits firmados
(rango $[-2048, +2047]$). La suma puede producir valores fuera de
este rango:

\begin{equation}
    r \in [-2048 + (-2048),\; +2047 + 2047] = [-4096,\; +4094]
    \label{eq:sum_range}
\end{equation}

\noindent
lo que requiere 13~bits con signo para representarse sin p\'{e}rdida.

\paragraph{Por qu\'{e} no truncar.}
Si se toma simplemente $r[11:0]$ (los 12~bits inferiores), un valor
como $+2100$ $(= \texttt{0x0834})$ se interpretar\'{\i}a
correctamente; pero $+2048$ $(= \texttt{0x0800})$ tiene el bit~11
a~1 y se interpretar\'{\i}a como $-2048$: una
\textbf{inversi\'{o}n catastr\'{o}fica de signo} que inyecta un error
de $4096$ unidades (la distancia m\'{a}xima posible). Esto distorsiona
severamente la constelaci\'{o}n y falsea los resultados de BER.

\paragraph{Soluci\'{o}n: saturaci\'{o}n.}
Se implementa aritm\'{e}tica de saturaci\'{o}n (\textit{clipping}):

\begin{equation}
    r_{\text{sat}}[n] = \begin{cases}
        +2047  & \text{si } r[n] > +2047 \\
        -2048  & \text{si } r[n] < -2048 \\
        r[n]   & \text{en otro caso}
    \end{cases}
    \label{eq:sat}
\end{equation}

En SystemVerilog, la extensi\'{o}n de signo y la comparaci\'{o}n se
realizan en 13~bits:

\begin{lstlisting}[caption={Saturaci\'{o}n aritm\'{e}tica en \texttt{channel\_top.sv}.},
                   label={lst:saturation}]
logic signed [DATA_WIDTH:0] sum_I;  // 13-bit signed
assign sum_I = {tx_I_d[2][DATA_WIDTH-1], tx_I_d[2]}  // sign-extend TX
             + {noise_I[DATA_WIDTH-1],   noise_I};    // sign-extend noise

always_comb begin
    if (sum_I > 13'sd2047)
        rx_I_sat = 12'sd2047;        // SAT_POS
    else if (sum_I < -13'sd2048)
        rx_I_sat = -12'sd2048;       // SAT_NEG
    else
        rx_I_sat = sum_I[11:0];      // Dentro de rango
end
\end{lstlisting}

\paragraph{Probabilidad de saturaci\'{o}n.}
Para $M = 128$ (SNR $\approx \SI{17}{\decibel}$), los niveles extremos
de la constelaci\'{o}n son $\pm 1943$ (margen de $2047 - 1943 = 104$
unidades). La probabilidad de que el ruido supere este margen es:

\begin{equation}
    P_{\text{sat}} = 2\,Q\!\left(\frac{104}{\sigma_w}\right)
    \label{eq:psat}
\end{equation}

\noindent
donde $Q(\cdot)$ es la funci\'{o}n Q gaussiana. Para $M = 128$,
$\sigma_w \approx 74$, de donde $P_{\text{sat}} \approx 2\,Q(1{,}4)
\approx 16\,\%$: la saturaci\'{o}n ya es frecuente, pero introduce
una compresi\'{o}n suave an\'{a}loga al recorte de un ADC real.
Para $M \leq 64$ (SNR $\geq \SI{23}{\decibel}$), $P_{\text{sat}} < 0{,}1\,\%$.

% ----------------------------------------------------------------------------
\subsection{Alineamiento de latencia y pipeline}
\label{subsec:latency}
% ----------------------------------------------------------------------------

El generador de ruido introduce 3~ciclos de latencia (LFSRs $\to$
suma registrada $\to$ escalado registrado). Para que la adici\'{o}n
$s[n] + w[n]$ sea coherente en el tiempo, la se\~{n}al TX se retarda
mediante una l\'{\i}nea de registros de 3~etapas:

\begin{equation}
    s_{\text{alineada}}[n] = s[n - 3]
    \label{eq:alignment}
\end{equation}

La suma se registra en una cuarta etapa, dando una latencia total del
bloque de canal de \textbf{4 ciclos de reloj}.

\begin{table}[H]
    \centering
    \caption{Latencia del subsistema de canal AWGN.}
    \label{tab:ch_latency}
    \begin{tabular}{@{} l c l @{}}
        \toprule
        \textbf{Etapa} & \textbf{Ciclos} & \textbf{Descripci\'{o}n} \\
        \midrule
        LFSRs y \'{a}rbol de sumas & 1 & Comb. + registro \\
        Centrado registrado          & 1 & Pipeline stage \\
        Escalado ($\times M$) + truncaci\'{o}n & 1 & Registro de salida \\
        Suma saturante + registro    & 1 & Salida final \\
        \midrule
        \textbf{Total}               & \textbf{4} & \\
        \bottomrule
    \end{tabular}
\end{table}


% ============================================================================
\section{Resultados de Validaci\'{o}n}
\label{sec:validation}
% ============================================================================

La validaci\'{o}n del canal AWGN se realiza mediante el testbench
\texttt{tb\_channel.sv}, que efect\'{u}a un barrido eparam\'{e}trico de
\texttt{noise\_magnitude} ($M \in \{0, 16, 64, 128, 255\}$) midiendo:

\begin{itemize}
    \item Estad\'{\i}sticas de las muestras de salida (media, m\'{\i}nimo,
          m\'{a}ximo) por nivel de ruido.
    \item Conteo de eventos de saturaci\'{o}n.
    \item Exportaci\'{o}n de las muestras $I/Q$ a fichero CSV para
          representaci\'{o}n gr\'{a}fica de la constelaci\'{o}n.
\end{itemize}

% ----------------------------------------------------------------------------
\subsection{Histograma de la distribuci\'{o}n de ruido}
\label{subsec:histogram}
% ----------------------------------------------------------------------------

La Figura~\ref{fig:histogram} muestra el histograma normalizado de
$10\,000$ muestras de ruido generadas con $M = 128$, superpuesto con la
PDF gaussiana te\'{o}rica de varianza equivalente. La concordancia
visual confirma la validez de la aproximaci\'{o}n CLT con $N = 16$.

\begin{figure}[H]
    \centering
    % ---- PLACEHOLDER: sustituir por la figura real ----
    \fbox{\parbox{0.7\textwidth}{\centering
        \vspace{3cm}
        \textit{Insertar histograma:} \texttt{docs/figures/awgn\_validation.png}
        \vspace{3cm}
    }}
    % Para insertar la figura real, descomentar:
    % \includegraphics[width=0.75\textwidth]{../figures/awgn_validation.png}
    \caption{Histograma del ruido generado ($M=128$, $N_{\text{muestras}}=10\,000$)
             frente a la PDF gaussiana te\'{o}rica $\mathcal{N}(0, \sigma_w^2)$.}
    \label{fig:histogram}
\end{figure}

% ----------------------------------------------------------------------------
\subsection{Dispersi\'{o}n de la constelaci\'{o}n}
\label{subsec:constellation}
% ----------------------------------------------------------------------------

La Figura~\ref{fig:constellation} ilustra c\'{o}mo la nube de puntos
$I/Q$ se expande al incrementar~$M$. Para $M = 0$ se observa la
constelaci\'{o}n ideal de 16~puntos; para $M = 255$ los cl\'{u}steres
se solapan, indicando una elevada probabilidad de error de s\'{\i}mbolo.

\begin{figure}[H]
    \centering
    \fbox{\parbox{0.7\textwidth}{\centering
        \vspace{3cm}
        \textit{Insertar mosaico de constelaciones para}
        $M \in \{0, 16, 64, 128, 255\}$
        \vspace{3cm}
    }}
    % \includegraphics[width=0.85\textwidth]{../figures/constellation_sweep.png}
    \caption{Constelaci\'{o}n 16-QAM recibida para distintos valores de
             \texttt{noise\_magnitude}.}
    \label{fig:constellation}
\end{figure}


% ============================================================================
\section{S\'{\i}ntesis de Decisiones de Dise\~{n}o}
\label{sec:decisions}
% ============================================================================

\begin{table}[H]
    \centering
    \caption{Resumen de decisiones de dise\~{n}o de la Fase~2 (Canal AWGN).}
    \label{tab:decisions}
    \small
    \begin{tabular}{@{} p{3.0cm} p{3.0cm} p{7.5cm} @{}}
        \toprule
        \textbf{Aspecto} & \textbf{Decisi\'{o}n} & \textbf{Justificaci\'{o}n} \\
        \midrule
        Generaci\'{o}n gaussiana &
        CLT ($N=16$) &
        Coste m\'{\i}nimo ($\sim$400~LUT, 0~BSRAM); latencia fija (3~ciclos);
        curtosis en exceso $< 0{,}08$. Descartados Box-Muller y Zigurat
        por alto consumo de recursos/latencia variable. \\[0.5em]

        Independencia de fuentes &
        16 polinomios primitivos verificados &
        11~anchuras \'{u}nicas + 5~rec\'{\i}procos; eliminadas las anchuras
        sin trinomio primitivo ($n \in \{16,19,24,26,27,30\}$).
        Referencia: XAPP052. \\[0.5em]

        Independencia I/Q &
        \texttt{INSTANCE\_ID} &
        Par\'{a}metro que modifica las 16~semillas de cada instancia;
        corrige el bug de correlaci\'{o}n I/Q detectado en revisi\'{o}n. \\[0.5em]

        Control de SNR &
        Registro \texttt{noise\_mag} [7:0] &
        Escalado lineal; $\text{SNR} \approx 59 - 20\log_{10}(M)$~dB;
        rango $\sim$11--$\infty$~dB. \\[0.5em]

        Gesti\'{o}n de overflow &
        Saturaci\'{o}n aritm\'{e}tica &
        Suma en 13~bits; recorte a $[-2048, +2047]$; evita inversi\'{o}n
        de signo catastr\'{o}fica de la truncaci\'{o}n simple. \\[0.5em]

        Alineamiento temporal &
        L\'{\i}nea de retardo 3~FF &
        Compensa la latencia del generador para suma coherente. \\
        \bottomrule
    \end{tabular}
\end{table}


% ============================================================================
% Referencias
% ============================================================================
\begin{thebibliography}{9}
    \bibitem{trinomial_table}
    R.~W. Watson,
    ``Table of primitive polynomials over GF(2) of degree up to 100,''
    \textit{Mathematics of Computation}, vol.~16, pp.~368--369, 1962.

    \bibitem{xapp052}
    P.~Alfke,
    ``Efficient Shift Registers, LFSR Counters, and Long Pseudo-Random
    Sequence Generators,''
    Xilinx Application Note XAPP052, v1.1, 1996.

    \bibitem{box_muller}
    G.~E.~P.~Box and M.~E.~Muller,
    ``A Note on the Generation of Random Normal Deviates,''
    \textit{The Annals of Mathematical Statistics}, vol.~29, no.~2,
    pp.~610--611, 1958.
\end{thebibliography}


% ============================================================================
\end{document}
